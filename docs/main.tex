% Options for packages loaded elsewhere
\PassOptionsToPackage{unicode}{hyperref}
\PassOptionsToPackage{hyphens}{url}
\PassOptionsToPackage{dvipsnames,svgnames*,x11names*}{xcolor}
%
\documentclass[
  nomag]{bxjsbook}
\usepackage{lmodern}
\usepackage{amssymb,amsmath}
\usepackage{ifxetex,ifluatex}
\ifnum 0\ifxetex 1\fi\ifluatex 1\fi=0 % if pdftex
  \usepackage[T1]{fontenc}
  \usepackage[utf8]{inputenc}
  \usepackage{textcomp} % provide euro and other symbols
\else % if luatex or xetex
  \usepackage{unicode-math}
  \defaultfontfeatures{Scale=MatchLowercase}
  \defaultfontfeatures[\rmfamily]{Ligatures=TeX,Scale=1}
  \setmainfont[]{DejaVu Serif}
  \setsansfont[]{DejaVu Sans}
  \setmonofont[]{Ricty}

% japanese font setting
% if the preset is specified
\ifxetex
  \usepackage[AutoFallBack=true]{zxjatype}
  \usepackage[noto,]{zxjafont}
  \usepackage{xeCJKfntef}
\fi
\ifluatex
  \usepackage[,noto]{luatexja-preset}
  \renewcommand{\kanjifamilydefault}{\gtdefault}
\fi

  \IfFileExists{pxrubrica.sty}{\usepackage{pxrubrica}}{}
\ifluatex
  \ltjsetparameter{%
    jacharrange={-2,-3},
    alxspmode={`/,allow},
    alxspmode={`#,allow},
    alxspmode={92,allow}
  }
\fi


% Use upquote if available, for straight quotes in verbatim environments
\IfFileExists{upquote.sty}{\usepackage{upquote}}{}
\IfFileExists{microtype.sty}{% use microtype if available
  \usepackage[]{microtype}
  \UseMicrotypeSet[protrusion]{basicmath} % disable protrusion for tt fonts
}{}
\makeatletter
\@ifundefined{KOMAClassName}{% if non-KOMA class
  \IfFileExists{parskip.sty}{%
    \usepackage{parskip}
  }{% else
    \setlength{\parindent}{0pt}
    \setlength{\parskip}{6pt plus 2pt minus 1pt}}
}{% if KOMA class
  \KOMAoptions{parskip=half}}
\makeatother
\usepackage{xcolor}
\IfFileExists{xurl.sty}{\usepackage{xurl}}{} % add URL line breaks if available
\IfFileExists{bookmark.sty}{\usepackage{bookmark}}{\usepackage{hyperref}}
\hypersetup{
  pdftitle={bookdown + rmdja による多様なファイル形式の日本語技術文書の作成},
  pdfauthor={Katagiri, Satoshi},
  colorlinks=true,
  linkcolor=blue,
  filecolor=Maroon,
  citecolor=blue,
  urlcolor=magenta,
  pdfcreator={LaTeX via pandoc}}
\urlstyle{same} % disable monospaced font for URLs
\usepackage{color}
\usepackage{fancyvrb}
\newcommand{\VerbBar}{|}
\newcommand{\VERB}{\Verb[commandchars=\\\{\}]}
\DefineVerbatimEnvironment{Highlighting}{Verbatim}{commandchars=\\\{\}}
% Add ',fontsize=\small' for more characters per line
\usepackage{framed}
\definecolor{shadecolor}{RGB}{248,248,248}
\newenvironment{Shaded}{\begin{snugshade}}{\end{snugshade}}
\newcommand{\AlertTok}[1]{\textcolor[rgb]{0.94,0.16,0.16}{#1}}
\newcommand{\AnnotationTok}[1]{\textcolor[rgb]{0.56,0.35,0.01}{\textbf{\textit{#1}}}}
\newcommand{\AttributeTok}[1]{\textcolor[rgb]{0.77,0.63,0.00}{#1}}
\newcommand{\BaseNTok}[1]{\textcolor[rgb]{0.00,0.00,0.81}{#1}}
\newcommand{\BuiltInTok}[1]{#1}
\newcommand{\CharTok}[1]{\textcolor[rgb]{0.31,0.60,0.02}{#1}}
\newcommand{\CommentTok}[1]{\textcolor[rgb]{0.56,0.35,0.01}{\textit{#1}}}
\newcommand{\CommentVarTok}[1]{\textcolor[rgb]{0.56,0.35,0.01}{\textbf{\textit{#1}}}}
\newcommand{\ConstantTok}[1]{\textcolor[rgb]{0.00,0.00,0.00}{#1}}
\newcommand{\ControlFlowTok}[1]{\textcolor[rgb]{0.13,0.29,0.53}{\textbf{#1}}}
\newcommand{\DataTypeTok}[1]{\textcolor[rgb]{0.13,0.29,0.53}{#1}}
\newcommand{\DecValTok}[1]{\textcolor[rgb]{0.00,0.00,0.81}{#1}}
\newcommand{\DocumentationTok}[1]{\textcolor[rgb]{0.56,0.35,0.01}{\textbf{\textit{#1}}}}
\newcommand{\ErrorTok}[1]{\textcolor[rgb]{0.64,0.00,0.00}{\textbf{#1}}}
\newcommand{\ExtensionTok}[1]{#1}
\newcommand{\FloatTok}[1]{\textcolor[rgb]{0.00,0.00,0.81}{#1}}
\newcommand{\FunctionTok}[1]{\textcolor[rgb]{0.00,0.00,0.00}{#1}}
\newcommand{\ImportTok}[1]{#1}
\newcommand{\InformationTok}[1]{\textcolor[rgb]{0.56,0.35,0.01}{\textbf{\textit{#1}}}}
\newcommand{\KeywordTok}[1]{\textcolor[rgb]{0.13,0.29,0.53}{\textbf{#1}}}
\newcommand{\NormalTok}[1]{#1}
\newcommand{\OperatorTok}[1]{\textcolor[rgb]{0.81,0.36,0.00}{\textbf{#1}}}
\newcommand{\OtherTok}[1]{\textcolor[rgb]{0.56,0.35,0.01}{#1}}
\newcommand{\PreprocessorTok}[1]{\textcolor[rgb]{0.56,0.35,0.01}{\textit{#1}}}
\newcommand{\RegionMarkerTok}[1]{#1}
\newcommand{\SpecialCharTok}[1]{\textcolor[rgb]{0.00,0.00,0.00}{#1}}
\newcommand{\SpecialStringTok}[1]{\textcolor[rgb]{0.31,0.60,0.02}{#1}}
\newcommand{\StringTok}[1]{\textcolor[rgb]{0.31,0.60,0.02}{#1}}
\newcommand{\VariableTok}[1]{\textcolor[rgb]{0.00,0.00,0.00}{#1}}
\newcommand{\VerbatimStringTok}[1]{\textcolor[rgb]{0.31,0.60,0.02}{#1}}
\newcommand{\WarningTok}[1]{\textcolor[rgb]{0.56,0.35,0.01}{\textbf{\textit{#1}}}}
\usepackage{longtable,booktabs,dcolumn}
% Correct order of tables after \paragraph or \subparagraph
\usepackage{etoolbox}
\makeatletter
\patchcmd\longtable{\par}{\if@noskipsec\mbox{}\fi\par}{}{}
\makeatother
% Allow footnotes in longtable head/foot
\IfFileExists{footnotehyper.sty}{\usepackage{footnotehyper}}{\usepackage{footnote}}
\makesavenoteenv{longtable}
\usepackage{graphicx,grffile}
\makeatletter
\def\maxwidth{\ifdim\Gin@nat@width>\linewidth\linewidth\else\Gin@nat@width\fi}
\def\maxheight{\ifdim\Gin@nat@height>\textheight\textheight\else\Gin@nat@height\fi}
\makeatother
% Scale images if necessary, so that they will not overflow the page
% margins by default, and it is still possible to overwrite the defaults
% using explicit options in \includegraphics[width, height, ...]{}
\setkeys{Gin}{width=\maxwidth,height=\maxheight,keepaspectratio}
% Set default figure placement to htbp
\makeatletter
\def\fps@figure{htbp}
\makeatother
\setlength{\emergencystretch}{3em} % prevent overfull lines
\providecommand{\tightlist}{%
  \setlength{\itemsep}{0pt}\setlength{\parskip}{0pt}}
\setcounter{secnumdepth}{5}
\usepackage[pdfbox]{gentombow}
\usepackage{booktabs}
\usepackage{longtable}
\usepackage{array}
\usepackage{multirow}
\usepackage{wrapfig}
\usepackage{float}
\usepackage{colortbl}
\usepackage{pdflscape}
\usepackage{tabu}
\usepackage{threeparttable}
\usepackage{threeparttablex}
\usepackage[normalem]{ulem}
\usepackage{makecell}
\usepackage[style=authoryear,]{biblatex}
\addbibresource{rmdja.bib}


\title{\texttt{bookdown} + \texttt{rmdja}
による多様なファイル形式の日本語技術文書の作成}
\author{Katagiri, Satoshi}
\date{2020-09-19}
\usepackage{bxtexlogo}
\colorlet{shadecolor}{gray!20}
\ifdefined\bibsection\renewcommand{\bibsection}{}\fi
\ifdefined\bibfont\renewcommand*{\bibfont}{\footnotesize}\fi
\usepackage{fmtcount}
\ifdefined\theFancyVerbLine\renewcommand{\theFancyVerbLine}{\small \padzeroes[2]{\decimal{FancyVerbLine}}}\fi % adjust row number position
\IfFileExists{bxcoloremoji.sty}{\usepackage{bxcoloremoji}}{}



%\renewcommand{\prepartname}{第}
%\renewcommand{\postpartname}{部}

% redefine quote format as modern
\makeatletter
\@ifpackageloaded{framed}{}{\usepackage{framed}}
\definecolor{quotebarcolor}{rgb}{0.2,0.2,0.2}
\renewenvironment{quote}{\def\FrameCommand{{\color{quotebarcolor}{\vrule width 3pt}}\hspace{10pt}}\MakeFramed{\advance\hsize-\width\FrameRestore}}{\endMakeFramed}
\makeatother

% ---- XeLaTeX 専用のあれ ----
\ifxetex
  \usepackage{letltxmacro}
  \setlength{\XeTeXLinkMargin}{1pt}
  \LetLtxMacro\SavedIncludeGraphics\includegraphics
  \def\includegraphics#1#{% #1 catches optional stuff (star/opt. arg.)
    \IncludeGraphicsAux{#1}%
  }%
  \newcommand*{\IncludeGraphicsAux}[2]{%
    \XeTeXLinkBox{%
      \SavedIncludeGraphics#1{#2}%
    }%
  }%
\fi

% --- 特殊ブロックをLaTeXで表現するためのマクロ ---
\makeatletter
\newenvironment{kframe}{%
\medskip{}
\setlength{\fboxsep}{.8em}
 \def\at@end@of@kframe{}%
 \ifinner\ifhmode%
  \def\at@end@of@kframe{\end{minipage}}%
  \begin{minipage}{\columnwidth}%
 \fi\fi%
 \def\FrameCommand##1{\hskip\@totalleftmargin \hskip-\fboxsep
 \colorbox{shadecolor}{##1}\hskip-\fboxsep
     % There is no \\@totalrightmargin, so:
     \hskip-\linewidth \hskip-\@totalleftmargin \hskip\columnwidth}%
 \MakeFramed {\advance\hsize-\width
   \@totalleftmargin\z@ \linewidth\hsize
   \@setminipage}}%
 {\par\unskip\endMakeFramed%
 \at@end@of@kframe}
\makeatother

\makeatletter
\@ifundefined{Shaded}{
}{\renewenvironment{Shaded}{\begin{kframe}}{\end{kframe}}}
\makeatother

\newenvironment{rmdblock}[1]
  {
  \begin{itemize}
  \renewcommand{\labelitemi}{
    \raisebox{-.7\height}[0pt][0pt]{
      {\setkeys{Gin}{width=3em,keepaspectratio}\includegraphics{_latex/_img/#1}}
    }
  }
  \setlength{\fboxsep}{1em}
  \begin{kframe}
  \item
  }
  {
  \end{kframe}
  \end{itemize}
  }
\newenvironment{rmdnote}
  {\begin{rmdblock}{note}}
  {\end{rmdblock}}
\newenvironment{rmdcaution}
  {\begin{rmdblock}{caution}}
  {\end{rmdblock}}
\newenvironment{rmdimportant}
  {\begin{rmdblock}{important}}
  {\end{rmdblock}}
\newenvironment{rmdtip}
  {\begin{rmdblock}{tip}}
  {\end{rmdblock}}
\newenvironment{rmdwarning}
  {\begin{rmdblock}{warning}}
  {\end{rmdblock}}

\usepackage{amsthm}
\newtheorem{theorem}{定理}[chapter]
\newtheorem{lemma}{補題}[chapter]
\newtheorem{corollary}{系}[chapter]
\newtheorem{proposition}{命題}[chapter]
\newtheorem{conjecture}{予想}[chapter]
\theoremstyle{definition}
\newtheorem{definition}{定義}[chapter]
\theoremstyle{definition}
\newtheorem{example}{例}[chapter]
\theoremstyle{definition}
\newtheorem{exercise}{練習}[chapter]
\theoremstyle{remark}
\newtheorem*{remark}{用例}
\newtheorem*{solution}{解答}
\begin{document}
\maketitle

\makeatletter
\def\emptypage@emptypage{%
    \hbox{}%
    \thispagestyle{headings}%
    \newpage%    
}%
\def\cleardoublepage{%
        \clearpage%
        \if@twoside%
            \ifodd\c@page%
                % do nothing
            \else%
                \emptypage@emptypage%
            \fi%
        \fi%
    }%
\makeatother

{
\hypersetup{linkcolor=}
\setcounter{tocdepth}{2}
\tableofcontents
}
\hypertarget{part-ux30a4ux30f3ux30c8ux30edux30c0ux30afux30b7ux30e7ux30f3}{%
\part{
イントロダクション}\label{part-ux30a4ux30f3ux30c8ux30edux30c0ux30afux30b7ux30e7ux30f3}}

\hypertarget{ux5e8fux6587}{%
\chapter*{序文}\label{ux5e8fux6587}}
\addcontentsline{toc}{chapter}{序文}

\href{https://creativecommons.org/licenses/by-nc/4.0/deed.ja}{\includegraphics{./40d24a2033dcf6947fd533dd28ecf6ed1a1dae6e.png}}

\textbf{注意: 絶賛作りかけ}

長大な技術文書や良質な技術文書を作成するには手間がかかる.
しかし時間をかければ良い文書になるわけではない. 無駄な手間を省き,
効率よく快適に文書を作成するべきである.

たとえばこういう経験はないだろうか.

\begin{itemize}
\tightlist
\item
  プログラムの解説のため,
  外部サービスでシンタックスハイライトしてもらったテキストを\textbf{コピーペーストで貼り付ける}
\item
  グラフや図解を専用アプリケーションで作成し貼り付ける.
  \textbf{修正のたびに貼り付け直す}
\item
  図表に言及する際に「図 1」「表 2」と\textbf{番号をタイプし,
  参照先へハイパーリンクを指定する}
\item
  本文中で引用した参考文献のリストを巻末に\textbf{コピーペーストし,
  過不足がないか目視で確認}
\item
  \(\sum_{k=1}^K\int_0^\infty f_k(x) dx\)
  などといった複雑な数式はプレーンテキストや HTML では表現できないため,
  \textbf{画像を生成して貼り付ける}
\item
  冒頭にかっこいいエピグラフを掲載したいので,
  \textbf{1時間かけて特別に枠やフォントを作成した}
\item
  市販のワードプロセッサで作成した文書を渡したら,
  \textbf{レイアウトが崩れて読めない}と言われた
\end{itemize}

本稿は文書作成者をこのような様々なブルシットから解放するのが目的である.

R Markdown (\texttt{rmarkdown}) は, R
プログラムを埋め込んだ動的なドキュメントから pandoc を利用して PDF や
HTML 形式の文書を作成するパッケージであり, 数式, 図表の挿入,
シンタックスハイライトされたプログラムなどを簡単な記述で掲載できる.
名前の通り, その基本構文は Markdown である. よって, Markdown と R
の知識が最低限あれば (R プログラムが必要ないなら Markdown だけでも)
文書を作成することができる.

\texttt{bookdown} パッケージは \texttt{rmarkdown} をもとに,
ページ数の多い文書を作成し, 配布するための機能を拡張したものである.
しかし, PDF
の出力に関しては欧文を前提としたフォーマットを使用しているため,
日本語の適切な表示 (組版やフォントの埋め込みなど)
のできる文書を作成するには高いハードルが存在した.

本稿では, \texttt{bookdown}
で日本語文書を作成する際の設定を容易にしたパッケージ \texttt{rmdja}
を利用した日本語技術文書の作成方法を解説する.

\hypertarget{ux672cux7a3fux306eux76eeux6a19}{%
\chapter*{本稿の目標}\label{ux672cux7a3fux306eux76eeux6a19}}
\addcontentsline{toc}{chapter}{本稿の目標}

\hypertarget{ux65e2ux5b58ux306eux30d5ux30a9ux30fcux30deux30c3ux30c8ux3068ux306eux9055ux3044}{%
\section*{既存のフォーマットとの違い}\label{ux65e2ux5b58ux306eux30d5ux30a9ux30fcux30deux30c3ux30c8ux3068ux306eux9055ux3044}}
\addcontentsline{toc}{section}{既存のフォーマットとの違い}

もちろん, 同様のことは既存のソフトウェアやサービスでも可能である.

たとえば はてなブログ, Qiita, といった既存のブログサービスには, Mathjax
による数式レンダリングやプログラムをシンタックスハイライトして表示する機能が最初から用意されているものもある.
しかしながら現状では以下のような制約がある.

\begin{itemize}
\tightlist
\item
  独自規格の構文が使いづらい, 一部本来と違う構文で数式を書く必要がある,
  ページ内リンクが使えない, など.
\item
  テキストエディタでしか書けない
\item
  \textbf{数十ページ相当のテキストを投稿しようとしただけ}でエラーが発生する.
\end{itemize}

また, \LaTeX (シンプルなテキストエディタでも, Overleaf や LyX
といった強力なエディタでも)
を普段使っている人間にとっては以下のような利点がある. R Markdown
はそもそも PDF 出力時は \LaTeX に依存しているため,
主な違いは操作の簡略化にある.

\begin{itemize}
\tightlist
\item
  外部プログラムで作成した画像や計算結果をコピーペーストせずそのまま貼り付けられる
\item
  LaTeX とほぼ同じ構文で数式を記入できる
\item
  主な設定は既に定義済みあり, 本文は簡易な Markdown で書くことができる,
  よって「TeXは複雑でわかりづらい,
  時代遅れのシステム」といった私怨混じりの批判を跳ね返せる
\end{itemize}

Word を普段使っている人間にとっては以下のような利点がある\footnote{ただし筆者は数年来
  Word を使っていないため,
  これらのいくつかは既に改善されているかもしれない.}.

\begin{itemize}
\tightlist
\item
  数十, 数百ページの文書を書いてもクラッシュすることがあまりない
\item
  輪郭のはっきりしたベクタ画像を簡単に貼り付けられる
\item
  図表の配置や相互参照を手動で書く必要がない
\item
  読み手の環境に依存してレイアウトが崩れにくいPDFファイルを出力できる
\end{itemize}

さらに, 作成した文書は PDF 形式で出力することはもちろん, HTML
形式で様々なサイトで掲載でき\footnote{\texttt{bookdown} 同様に R
  Markdown で作成した文書をブログ風のフォーマットで出力する
  \texttt{blogdown} パッケージというものも存在する.}たり,
電子書籍ファイルとしても出力可能である.
このような多様な出力形式への対応しているソフトウェアはあまり例を見ない.

\texttt{bookdown} はこのように便利で,
公式ドキュメントがとても充実しているにも関わらず,
日本語に適したレイアウトの設定の煩雑さからあまり普及していない\footnote{前例として
  bookdown
  で作成した文書を技術書展で配布している人が書いたブログが存在する:
  \url{https://teastat.blogspot.com/2019/01/bookdown.html}}.

\begin{itemize}
\tightlist
\item
  ``\href{https://rmarkdown.rstudio.com/docs/index.html}{Dynamic
  Documents for R・rmarkdown}''
\item
  ``\href{https://github.com/rstudio/bookdown-demo}{bookdown demo}''
\item
  ``\href{https://bookdown.org/yihui/bookdown/}{\emph{bookdown:
  Authoring Books and Technical Documents with R Markdown}}''
\item
  ``\href{https://bookdown.org/yihui/rmarkdown/}{\emph{R Markdown
  Definiteive Guide}}''
\item
  ``\href{https://bookdown.org/yihui/rmarkdown-cookbook}{\emph{R
  Markdown Cookbook}}''\footnote{2020/10/19
    に書籍としても発売されるらしい.}
\end{itemize}

基本的なことがらの多くは上記を読めば分かるのでここでは基本機能をダイジェストで伝えた上で,
これらの資料に書いてない応用技を紹介する. YAML
のオプションの意味についてはソースコードにコメントを書いた. 以下, 単に,
**BKD* と書けば ``\emph{\texttt{bookdown}: Authoring Books and Technical
Documents with R Markdown}'' \autocite{R-bookdown} を, RDG と書けば
``\emph{R Markdown: The Definitive GUide}'' \autocite{rmarkdown2018} を,
RCB と書けば ``\emph{R Markdown Cookbook}'' \autocite{xie2020Markdown}
を指すことにする.

さらに, Python の \texttt{jupyter} は Python
のコードチャンクとその結果を簡単に表示できる文書作成ツールである.
出力オプションの少なさ (たとえば長大なコードもそのまま掲載されてしまう)
や, IDE として見ても機能が少ないことからあまり使い勝手がよくなかったが,
最近登場した \textbf{Jupyter book} はドキュメント生成能力を強化している.
しかし \texttt{R\ Markdown}/\texttt{bookdown}
ほどPDF形式のことは考慮していないように見える.

\hypertarget{ux6614ux8a71ux3042ux308bux3044ux306fux65e2ux5b58ux8cc7ux6599ux3068ux306eux9055ux3044}{%
\section*{昔話あるいは既存資料との違い}\label{ux6614ux8a71ux3042ux308bux3044ux306fux65e2ux5b58ux8cc7ux6599ux3068ux306eux9055ux3044}}
\addcontentsline{toc}{section}{昔話あるいは既存資料との違い}

日本語コミュニティにおいて \texttt{bookdown} は以前から言及されていた.
例えば 2016 年の kazutan
氏のスライド『\href{https://kazutan.github.io/JapanR2016/JapanR2016.html\#/}{Rで本を作りたい}』があり,
同じく \texttt{bookdown} 製のデモページ,
『\href{https://kazutan.github.io/bookdown_test/hoge.html}{Bookdownを用いた図表番号の自動付与と参照のテスト}』がある\footnote{ソースはこちら:
  \url{https://kazutan.github.io/bookdown_test/hoge.html}}.

それ以降も R Markdown に関する情報を発信する人はいたが,
大きく話題になることが少なかった. ユーザはいるものの,
もっぱらHTMLへの出力用途に使い, PDF
や同時出力に挑戦する人間はほとんど見られなかった.
普及していない理由は情報の絶対的な少なさ (そして古さ) にある.
本稿の目標はこのような状況を改善することにある.

もしすでに R Markdown や Bookdown に触れて,
ネット上の情報を元に文書を作った人に対しては,
本稿を読むことで以下のようなメリットがあるかもしれない

\begin{itemize}
\tightlist
\item
  IPA フォント以外のフォントを指定する方法
\item
  表のスタイル, 画像の埋め込み方, 見出しのスタイル,
  といった基本的なレイアウトを見やすく変更する方法
\end{itemize}

なぜかネット上のこの手の情報では IPA フォントを使いたがる例が多い.
10年前ならいざ知らず,
もうほとんどの主要OSではIPAフォントはプリインストールされてないのでこだわる理由はあまりないと思うのだが,
どうだろうか? ノスタルジーに浸るのはもちろん自由だ,
むしろファイルを持ってさえいれば\textbf{IPAモナーフォントを埋め込むことも可能}だろう.
しかしフリーフォントならむしろカバレッジに優れる Noto
フォントが便利だし,
あるいは\textbf{游書体}とか\textbf{ヒラギノ}とか各OSの基本フォントを使いたいだろう.
フォントにこだわる人間なら,
欧文と和文で異なるフォントを使う\textbf{混植}がしたい人もいるだろう.
\texttt{rmdja} では, 既に公開した \texttt{pdf\_presentation\_ja}
フォーマットと同様に, \texttt{bookdown} でも混植できる\footnote{ただし,
  現時点では HTML に対してフォントを細かく設定する機能, および HTML と
  PDF のフォントを一致させる機能はない.
  これは技術的というよりライセンス的に制約が多いからだ.}.

実際に変えようとすると, それでも結構ややこしいことがわかる.
例えば以下のエントリ

\url{https://notchained.hatenablog.com/entry/2018/08/12/140637}

\begin{enumerate}
\def\labelenumi{\arabic{enumi}.}
\tightlist
\item
  R Markdown では通常, \LaTeX は \texttt{tinytex}
  でインストールしたものが呼び出される
\item
  そして \texttt{tinytex} は \pLaTeX や p\BibTeX などを想定していない.
\item
  その範囲で実行するには, \XeLaTeX または \LuaLaTeX が必要
\item
  しかし, デフォルトでは文書クラスが日本語文書向けでないので,
  レイアウトがあまりよろしくない. 具体的には禁則処理がおかしいとか,
  見出しが英文風だとか.
\end{enumerate}

PDFでも日本語を表示する最低限の設定は, YAML フロントマターだけで行える.
これは Atusy
氏が『\href{https://blog.atusy.net/2019/05/14/rmd2pdf-any-font/}{R
Markdown + XeLaTeX で日本語含め好きなフォントを使って PDF
を出力する}』で紹介しているものとだいたい同じ方法になる.

\begin{Shaded}
\begin{Highlighting}[]
\FunctionTok{output}\KeywordTok{:}
\AttributeTok{  }\FunctionTok{pdf\_document}\KeywordTok{:}
\AttributeTok{    }\FunctionTok{latex\_engine}\KeywordTok{:}\AttributeTok{ xelatex}
\FunctionTok{header{-}includes}\KeywordTok{: }\CharTok{|}
\NormalTok{  \textbackslash{}setCJKmonofont\{Noto Sans Mono CJK JP\}}
\NormalTok{  \textbackslash{}setCJKsansfont\{Noto Sans CJK JP\}}
\FunctionTok{documentclass}\KeywordTok{:}\AttributeTok{ bxjsreport}
\FunctionTok{mainfont}\KeywordTok{:}\AttributeTok{ Noto Serif}
\FunctionTok{sansfont}\KeywordTok{:}\AttributeTok{ Noto Serif}
\FunctionTok{monofont}\KeywordTok{:}\AttributeTok{ Noto Mono}
\FunctionTok{CJKmainfont}\KeywordTok{:}\AttributeTok{ Noto Serif CJK JP}
\end{Highlighting}
\end{Shaded}

このままでも, とりあえず文字化けすることなく日本語を表示できる.
しかし文書として整ったものにするのは難しい

このままでは参考文献リストの表示も不自然なままである.
だがこれ以上のカスタマイズは, Atusy
氏がやっているようにテンプレートを修正する必要がある.

なるべく選択肢は広げておきたいが,
なんでもありではかえって余計なことをしがちである. よって既に作成した
beamer フォーマットと同様に, \XeLaTeX および \LuaLaTeX
のみの対応を想定している.

\hypertarget{ux60f3ux5b9aux8aadux8005}{%
\section*{想定読者}\label{ux60f3ux5b9aux8aadux8005}}
\addcontentsline{toc}{section}{想定読者}

既に紹介したように, R を普段使わない人間でも \texttt{bookdown}
で同人技術書を執筆したという事例もある.
よって非Rユーザにもある程度配慮して書いておくが,
あまりに細かいRの仕様説明などは行わない.
\textbf{良い参考書が既に数多く存在する}からだ. よって 非Rユーザは,
本稿にある使用例のうち,
自分の書きたいものに使えそうなのものをつまみ食いして使用することになるだろう.
本稿ではそのような使い方をするのに最低限必要なセットアップの知識のみ記載する.

\hypertarget{r-markdown-ux30e6ux30fcux30b6ux5411ux3051ux306eux30e1ux30eaux30c3ux30c8}{%
\section*{R Markdown
ユーザ向けのメリット}\label{r-markdown-ux30e6ux30fcux30b6ux5411ux3051ux306eux30e1ux30eaux30c3ux30c8}}
\addcontentsline{toc}{section}{R Markdown ユーザ向けのメリット}

R Markdown を使ったことのある人ならわかるろうが, HTML
を作るのに納得の行く \texttt{output} の指定をしていたらこうなった

\begin{Shaded}
\begin{Highlighting}[]
\AttributeTok{bookdown:}\FunctionTok{:gitbook}\KeywordTok{:}
\AttributeTok{  }\FunctionTok{split\_by}\KeywordTok{:}\AttributeTok{ chapter}
\AttributeTok{  }\FunctionTok{dev}\KeywordTok{:}\AttributeTok{ png}
\AttributeTok{  }\FunctionTok{dev{-}args}\KeywordTok{:}
\AttributeTok{    }\FunctionTok{res}\KeywordTok{:}\AttributeTok{ }\DecValTok{200}
\AttributeTok{  }\FunctionTok{css}\KeywordTok{:}\AttributeTok{ }
\AttributeTok{    }\KeywordTok{{-}}\AttributeTok{ }\StringTok{\textquotesingle{}../../styles/css/style.css\textquotesingle{}}
\AttributeTok{    }\KeywordTok{{-}}\AttributeTok{ }\StringTok{\textquotesingle{}../../styles/css/toc.css\textquotesingle{}}
\AttributeTok{  }\FunctionTok{keep\_md}\KeywordTok{:}\AttributeTok{ }\CharTok{true}
\AttributeTok{  }\FunctionTok{config}\KeywordTok{:}
\AttributeTok{    }\FunctionTok{toc}\KeywordTok{:}
\AttributeTok{      }\FunctionTok{collapse}\KeywordTok{:}\AttributeTok{ none}
\FunctionTok{      before}\KeywordTok{: }\CharTok{|}
\NormalTok{        \textless{}li\textgreater{}\textless{}a href="./"\textgreater{}Top\textless{}/a\textgreater{}\textless{}/li\textgreater{}}
\FunctionTok{      after}\KeywordTok{: }\CharTok{|}
\NormalTok{        \textless{}li\textgreater{}\textless{}a href="https://bookdown.org" target="\_blank"\textgreater{}Published with bookdown\textless{}/a\textgreater{}\textless{}/li\textgreater{}}
\AttributeTok{    }\FunctionTok{toolbar}\KeywordTok{:}
\AttributeTok{      }\FunctionTok{position}\KeywordTok{:}\AttributeTok{ fixed}
\AttributeTok{    }\FunctionTok{edit }\KeywordTok{:}\AttributeTok{ }\CharTok{null}
\AttributeTok{    }\FunctionTok{download}\KeywordTok{:}
\AttributeTok{      }\KeywordTok{{-}}\AttributeTok{ pdf}
\AttributeTok{      }\KeywordTok{{-}}\AttributeTok{ epub}
\AttributeTok{      }\KeywordTok{{-}}\AttributeTok{ mobi}
\AttributeTok{    }\FunctionTok{fontsettings}\KeywordTok{:}
\AttributeTok{      }\FunctionTok{theme}\KeywordTok{:}\AttributeTok{ white}
\AttributeTok{      }\FunctionTok{family}\KeywordTok{:}\AttributeTok{ serif}
\AttributeTok{    }\FunctionTok{sharing}\KeywordTok{:}
\AttributeTok{      }\FunctionTok{github}\KeywordTok{:}\AttributeTok{ }\CharTok{true}
\AttributeTok{      }\FunctionTok{twitter}\KeywordTok{:}\AttributeTok{ }\CharTok{true}
\AttributeTok{      }\FunctionTok{facebook}\KeywordTok{:}\AttributeTok{ }\CharTok{true}
\AttributeTok{      }\FunctionTok{all}\KeywordTok{:}\AttributeTok{ }\KeywordTok{[}\StringTok{\textquotesingle{}linkedin\textquotesingle{}}\KeywordTok{,}\AttributeTok{ }\StringTok{\textquotesingle{}vk\textquotesingle{}}\KeywordTok{,}\AttributeTok{ }\StringTok{\textquotesingle{}weibo\textquotesingle{}}\KeywordTok{,}\AttributeTok{ }\StringTok{\textquotesingle{}instapaper\textquotesingle{}}\KeywordTok{]}
\end{Highlighting}
\end{Shaded}

しかし \texttt{rmdja} ではこうだ.

\begin{Shaded}
\begin{Highlighting}[]
\AttributeTok{rmdja:}\FunctionTok{:gitbook\_ja}\KeywordTok{:}
\AttributeTok{  }\FunctionTok{keep\_md}\KeywordTok{:}\AttributeTok{ }\CharTok{true}
\end{Highlighting}
\end{Shaded}

PDF の場合も紹介しよう.

\begin{Shaded}
\begin{Highlighting}[]
\AttributeTok{bookdown:}\FunctionTok{:pdf\_book}\KeywordTok{:}
\AttributeTok{  }\FunctionTok{toc\_depth}\KeywordTok{:}\AttributeTok{ }\DecValTok{3}
\AttributeTok{  }\FunctionTok{toc\_appendix}\KeywordTok{:}\AttributeTok{ }\CharTok{true}
\AttributeTok{  }\FunctionTok{toc\_bib}\KeywordTok{:}\AttributeTok{ }\CharTok{true}
\AttributeTok{  }\FunctionTok{latex\_engine}\KeywordTok{:}\AttributeTok{ xelatex}
\AttributeTok{  }\FunctionTok{keep\_tex}\KeywordTok{:}\AttributeTok{ }\CharTok{true}
\AttributeTok{  }\FunctionTok{keep\_md}\KeywordTok{:}\AttributeTok{ }\CharTok{true}
\AttributeTok{  }\FunctionTok{citation\_package}\KeywordTok{:}\AttributeTok{ natbib}
\AttributeTok{  }\FunctionTok{pandoc\_args}\KeywordTok{:}
\AttributeTok{    }\KeywordTok{{-}}\AttributeTok{ }\StringTok{\textquotesingle{}{-}{-}top{-}level{-}division=chapter\textquotesingle{}}
\AttributeTok{    }\KeywordTok{{-}}\AttributeTok{ }\StringTok{\textquotesingle{}{-}{-}extract{-}media\textquotesingle{}}
\AttributeTok{    }\KeywordTok{{-}}\AttributeTok{ }\StringTok{\textquotesingle{}.\textquotesingle{}}
\AttributeTok{  }\FunctionTok{template}\KeywordTok{:}\AttributeTok{ XXXXX.tex.template\textquotesingle{}}
\AttributeTok{  }\FunctionTok{dev}\KeywordTok{:}\AttributeTok{ }\StringTok{"cairo\_pdf"}
\AttributeTok{  }\FunctionTok{out\_width}\KeywordTok{:}\AttributeTok{ }\StringTok{"100\%"}
\AttributeTok{  }\FunctionTok{out\_height}\KeywordTok{:}\AttributeTok{ }\StringTok{"100\%"}
\AttributeTok{  }\FunctionTok{quote\_footer}\KeywordTok{:}\AttributeTok{ }\KeywordTok{[}\StringTok{"}\SpecialCharTok{\textbackslash{}\textbackslash{}}\StringTok{VA\{"}\KeywordTok{,}\AttributeTok{ }\StringTok{"\}\{\}"}\KeywordTok{]}
\AttributeTok{  }\FunctionTok{extra\_dependencies}\KeywordTok{:}\AttributeTok{ gentombow}
\end{Highlighting}
\end{Shaded}

こうなる

\begin{Shaded}
\begin{Highlighting}[]
\AttributeTok{rmdja:}\FunctionTok{:pdf\_book\_ja}\KeywordTok{:}
\AttributeTok{  }\FunctionTok{keep\_tex}\KeywordTok{:}\AttributeTok{ }\CharTok{true}
\AttributeTok{  }\FunctionTok{keep\_md}\KeywordTok{:}\AttributeTok{ }\CharTok{true}
\AttributeTok{  }\FunctionTok{tombow}\KeywordTok{:}\AttributeTok{ }\CharTok{true}
\end{Highlighting}
\end{Shaded}

さらにチャンクオプションや場合によっては .tex
ファイルのテンプレートすら調整する必要もあった.
これらは和文と欧文の組版の違いに由来するものである.
これらの煩雑な設定を内蔵し,
かつフォントの設定など環境に応じて変更する必要のあるものをある程度\textbf{自動的に設定}するようにしている.

\hypertarget{part-ux6700ux4f4eux9650ux306eux30c1ux30e5ux30fcux30c8ux30eaux30a2ux30eb}{%
\part{
最低限のチュートリアル}\label{part-ux6700ux4f4eux9650ux306eux30c1ux30e5ux30fcux30c8ux30eaux30a2ux30eb}}

\hypertarget{ux6e96ux5099}{%
\chapter{準備}\label{ux6e96ux5099}}

\hypertarget{ux30a4ux30f3ux30b9ux30c8ux30fcux30eb}{%
\section{インストール}\label{ux30a4ux30f3ux30b9ux30c8ux30fcux30eb}}

文書を生成するのに必要なものをインストールする.

このドキュメントは \texttt{rmdja} パッケージに含まれている.
よってまずはこれをダウンロードしてほしい.

ただし, \texttt{bookdown} ver. 0.2
時点ではソースファイルのフォルダ配置していに不具合があるため{[}\^{}bookdown-subdir-error{]},
もし0.2以前のバージョンを使用しているなら, 更新するか, github
から開発版をインストールする. 同様に, \texttt{rmarkdown} も CRAN
ではなく開発版をインストールしたほうがよい.

\begin{Shaded}
\begin{Highlighting}[numbers=left,,]
\NormalTok{remotes}\OperatorTok{::}\KeywordTok{install\_github}\NormalTok{(}\StringTok{"rstudio/rmarkdown"}\NormalTok{)}
\NormalTok{remotes}\OperatorTok{::}\KeywordTok{install\_github}\NormalTok{(}\StringTok{"rstudio/bookdown"}\NormalTok{)}
\end{Highlighting}
\end{Shaded}

次に,
最低限のファイルやパッケージで動くほうのデモ用ディレクトリをコピーする

\begin{Shaded}
\begin{Highlighting}[numbers=left,,]
\KeywordTok{file.copy}\NormalTok{(}\KeywordTok{system.file}\NormalTok{(}\StringTok{"resources/examples/bookdown{-}minimal"}\NormalTok{, }\DataTypeTok{package =} \StringTok{"rmdja"}\NormalTok{), }\StringTok{"./"}\NormalTok{, }\DataTypeTok{recursive =}\NormalTok{ T)}
\end{Highlighting}
\end{Shaded}

\begin{verbatim}
[1] TRUE
\end{verbatim}

bookdown の文書生成は従来の R Markdown と違い, RStudio の \texttt{knit}
ボタンでは\textbf{できない}. 代わりに, 以下の2通りの方法がある.

\begin{enumerate}
\def\labelenumi{\arabic{enumi}.}
\tightlist
\item
  \texttt{bookdown::render\_book(\textquotesingle{}index.Rmd\textquotesingle{},\ format\ =\ "bookdown::gitbook")}
  などを呼び出す
\item
  RStudio の Build ペーンを使う
\end{enumerate}

前者は, 以下のように実行する. 順にHTML, PDF, epub を出力している

\begin{Shaded}
\begin{Highlighting}[numbers=left,,]
\NormalTok{bookdown}\OperatorTok{::}\KeywordTok{render\_book}\NormalTok{(}\StringTok{"index.Rmd"}\NormalTok{, }\StringTok{"rmdja::gitbook\_ja"}\NormalTok{)}
\NormalTok{bookdown}\OperatorTok{::}\KeywordTok{render\_book}\NormalTok{(}\StringTok{"index.Rmd"}\NormalTok{, }\StringTok{"rmdja::pdf\_book\_ja"}\NormalTok{)}
\NormalTok{bookdown}\OperatorTok{::}\KeywordTok{render\_book}\NormalTok{(}\StringTok{"index.Rmd"}\NormalTok{, }\StringTok{"bookdown::epub\_book"}\NormalTok{)}
\end{Highlighting}
\end{Shaded}

コピーしたディレクトリ \texttt{bookdown-minimal} を設定する (図
\ref{fig:build-pane1-1}, \ref{fig:build-pane1-2}).

\begin{figure}

{\centering \includegraphics[width=1\linewidth,height=1\textheight,keepaspectratio]{img//build-pane} 

}

\caption{Build ペーンの手動設定}\label{fig:build-pane1-1}
\end{figure}
\begin{figure}

{\centering \includegraphics[width=1\linewidth,height=1\textheight,keepaspectratio]{img//build-pane-build} 

}

\caption{Build ペーンの手動設定}\label{fig:build-pane1-2}
\end{figure}

これで \texttt{\_book} フォルダに出力がされる.

\hypertarget{part-r-markdown-ux3068-bookdown-ux306eux57faux672cux6a5fux80fd}{%
\part{R Markdown と Bookdown
の基本機能}\label{part-r-markdown-ux3068-bookdown-ux306eux57faux672cux6a5fux80fd}}

\hypertarget{ux3053ux306eux90e8ux306eux6982ux8981}{%
\chapter*{この部の概要}\label{ux3053ux306eux90e8ux306eux6982ux8981}}
\addcontentsline{toc}{chapter}{この部の概要}

ここではまず, R Markdown の基本的な機能を紹介する. つまり
\texttt{bookdown} 特有のものではなく, R Markdown
全般で使用できる機能も含めて紹介する.
これ以降は自己言及的な説明が多いため,
この文書を生成しているソースコードと比較しながら確認することをおすすめする.
ここで紹介する機能は BKD, RDG, RCB での記述に基づく.
これら3つのドキュメントを読めば, ほとんどのことは可能になる ---
\texttt{rmdja} を作る理由になった LaTeX テンプレートの修正以外は ---
のだが,
本稿の重要な目的の1つは\textbf{複数のファイル形式を両立すること}であるので,
それができない書き方には触れないし,
技術文書の作成にあまり使わないような機能の動作確認はおこなわず,
技術文書作成で頻繁に使われ, 便利と思える機能のみ紹介する.

どちらにしろそのうちこれらを翻訳してくれる人が現れることだろう\ldots たぶん.

\hypertarget{ux9759ux7684ux306aux30b3ux30f3ux30c6ux30f3ux30c4ux306eux4f5cux6210}{%
\chapter{静的なコンテンツの作成}\label{ux9759ux7684ux306aux30b3ux30f3ux30c6ux30f3ux30c4ux306eux4f5cux6210}}

日本語で書かれた資料でごく基本的なことについて,
『\href{https://kazutan.github.io/kazutanR/Rmd_intro.html}{R
Markdown入門}』で一通り紹介されている. やや応用的なことも 『R Markdown
ユーザーののための Pandoc's Markdown』に書かれている.

また, 既に作成している beamer
の用例ファイルもどのようなことができるかの参考になるだろう.
ただしこちらは PDF のみの出力を前提としているため, 一部の機能は HTML
で使うことができない.

まずは, 単なるマークアップ,
つまりプログラミングの複雑な処理を考えなくても良いタイプの構文を紹介する.
それらの多くは単なる Markdown のものと同じである.

\hypertarget{markdown-ux306eux57faux672cux69cbux6587}{%
\section{Markdown
の基本構文}\label{markdown-ux306eux57faux672cux69cbux6587}}

一応基本の Markdown の構文も挙げておく.
詳細は(ref:BKDB)``\href{https://bookdown.org/yihui/bookdown/markdown-syntax.html}{Ch.
2.2 Markdown Syntax}'' を参照.

\hypertarget{ux30a4ux30f3ux30e9ux30a4ux30f3ux3067ux306eux66f8ux5f0fux5909ux66f4}{%
\subsection{インラインでの書式変更}\label{ux30a4ux30f3ux30e9ux30a4ux30f3ux3067ux306eux66f8ux5f0fux5909ux66f4}}

テキストの一部のみ書式を変える

アンダースコアで強調 (イタリック)

\begin{Shaded}
\begin{Highlighting}[]
\NormalTok{\_underscore\_}
\end{Highlighting}
\end{Shaded}

\emph{underscore}

\texttt{**} 2つで太字強調

\begin{Shaded}
\begin{Highlighting}[]
\NormalTok{**太字強調**}
\end{Highlighting}
\end{Shaded}

\textbf{太字強調}

等幅フォント

\begin{Shaded}
\begin{Highlighting}[]
\InformationTok{\textasciigrave{}bookdown\textasciigrave{}}\NormalTok{ と }\InformationTok{\textasciigrave{}rmdja\textasciigrave{}}
\end{Highlighting}
\end{Shaded}

\texttt{bookdown} と \texttt{rmdja}

本文中に入力した URL は自動判別され, ハイパーリンクが付けられる. また,
\texttt{{[}テキスト{]}(URL)} という書式で,
テキストに対してハイパーリンクを付けることができる.

\begin{Shaded}
\begin{Highlighting}[]

\NormalTok{URL は自動判別される: https://github.com/Gedevan{-}Aleksizde/my\_latex\_templates/tree/master/rmdja}

\CommentTok{[}\OtherTok{\textasciigrave{}rmdja\textasciigrave{} の github リポジトリ}\CommentTok{](https://github.com/Gedevan{-}Aleksizde/my\_latex\_templates/tree/master/rmdja)}
\end{Highlighting}
\end{Shaded}

URL は自動判別される:
\url{https://github.com/Gedevan-Aleksizde/my_latex_templates/tree/master/rmdja}

\href{https://github.com/Gedevan-Aleksizde/my_latex_templates/tree/master/rmdja}{\texttt{rmdja}
の github リポジトリ}

\hypertarget{ux30d6ux30edux30c3ux30afux8981ux7d20}{%
\subsection{ブロック要素}\label{ux30d6ux30edux30c3ux30afux8981ux7d20}}

以降は行内では使えず,
\textbf{適切に表示するには前後に改行を挟む必要}のあるタイプの構文である.

まず, 引用ブロックを使えばかっこいいエピグラフを書き放題である.

\begin{Shaded}
\begin{Highlighting}[]
\AttributeTok{\textgreater{} Нужны новые формы. Новые формы нужны, а если их нет, то лучше ничего не нужно.}
\AttributeTok{\textgreater{}}
\AttributeTok{\textgreater{} 新しいフォーマットが必要なんですよ. 新しいフォーマットが. それがないというなら, いっそ何もないほうがいい. }
\AttributeTok{\textgreater{}}
\AttributeTok{\textgreater{} }\InformationTok{\textasciigrave{}\textbackslash{}r tufte::quote\_footer(\textquotesingle{}{-}{-}{-} A. チェーホフ『かもめ』\textquotesingle{})\textasciigrave{}}
\end{Highlighting}
\end{Shaded}

\begin{quote}
Нужны новые формы. Новые формы нужны, а если их нет, то лучше ничего не
нужно.

新しいフォーマットが必要なんですよ. 新しいフォーマットが.
それがないというなら, いっそ何もないほうがいい.

\hfill --- A. チェーホフ『かもめ』
\end{quote}

\texttt{rmdja} では, HTML と PDF
両方で同様のデザインの枠で表示するようにしている.

Markdown では \texttt{\#} は見出しを意味するが, \texttt{bookdown}
にはさらにオプションが用意されている.

\texttt{\#\ 見出し名\ \{-\}} で,
セクション番号のつかない見出しを用意できる. 序文, 章末の参考文献,
付録のセクションに使えるだろう. さらに, \texttt{bookdown} では
\texttt{\#\ (PART)\ 見出し名} で「部」の見出しを作ることができる.
この見出しは セクションの合間に挟まるが, 選択することはできない.
文書が長くなったときに, より大きな区切りを付けるのに役に立つだろう.
さらに, \texttt{\#\ (APPENDIX)\ 見出し名\ \{-\}} で, 以降の見出しの頭に
「補遺 A, B, C, \ldots」と付番できる.

箇条書きは以下のように書ける.

\begin{Shaded}
\begin{Highlighting}[]
\SpecialStringTok{* }\NormalTok{iris setosa}
\SpecialStringTok{* }\NormalTok{iris versicolor}
\SpecialStringTok{* }\NormalTok{iris virginica}
\end{Highlighting}
\end{Shaded}

\begin{itemize}
\tightlist
\item
  iris setosa
\item
  iris versicolor
\item
  iris virginica
\end{itemize}

\begin{Shaded}
\begin{Highlighting}[]
\SpecialStringTok{1. }\NormalTok{iris setosa}
\SpecialStringTok{2. }\NormalTok{iris versicolor}
\SpecialStringTok{3. }\NormalTok{iris virginica}
\end{Highlighting}
\end{Shaded}

\begin{enumerate}
\def\labelenumi{\arabic{enumi}.}
\tightlist
\item
  iris setosa
\item
  iris versicolor
\item
  iris virginica
\end{enumerate}

インデントを使えばネストできる.

\begin{itemize}
\tightlist
\item
  課長

  \begin{itemize}
  \tightlist
  \item
    課長補佐

    \begin{itemize}
    \tightlist
    \item
      課長補佐代理

      \begin{itemize}
      \tightlist
      \item
        課長補佐代理心得
      \end{itemize}
    \end{itemize}
  \end{itemize}
\end{itemize}

\hypertarget{markdown-ux3092ux4f7fux3063ux305fux56f3ux8868ux306eux633fux5165}{%
\section{Markdown
を使った図表の挿入}\label{markdown-ux3092ux4f7fux3063ux305fux56f3ux8868ux306eux633fux5165}}

markdown は表を記入することもできる.

\begin{Shaded}
\begin{Highlighting}[]
\AnnotationTok{Table:}\CommentTok{ Markdown 記法の表}

\NormalTok{ Sepal.Length   Sepal.Width   Petal.Length   Petal.Width}
\NormalTok{{-}{-}{-}{-}{-}{-}{-}{-}{-}{-}{-}{-}{-}  {-}{-}{-}{-}{-}{-}{-}{-}{-}{-}{-}{-}  {-}{-}{-}{-}{-}{-}{-}{-}{-}{-}{-}{-}{-}  {-}{-}{-}{-}{-}{-}{-}{-}{-}{-}{-}{-}}
\InformationTok{          5.1           3.5            1.4           0.2}
\InformationTok{          4.9           3.0            1.4           0.2}
\InformationTok{          4.7           3.2            1.3           0.2}
\InformationTok{          4.6           3.1            1.5           0.2}
\InformationTok{          5.0           3.6            1.4           0.2}
\InformationTok{          5.4           3.9            1.7           0.4}
\end{Highlighting}
\end{Shaded}

\begin{longtable}[]{@{}rrrr@{}}
\caption{Markdown 記法の表}\tabularnewline
\toprule
Sepal.Length & Sepal.Width & Petal.Length & Petal.Width\tabularnewline
\midrule
\endfirsthead
\toprule
Sepal.Length & Sepal.Width & Petal.Length & Petal.Width\tabularnewline
\midrule
\endhead
5.1 & 3.5 & 1.4 & 0.2\tabularnewline
4.9 & 3.0 & 1.4 & 0.2\tabularnewline
4.7 & 3.2 & 1.3 & 0.2\tabularnewline
4.6 & 3.1 & 1.5 & 0.2\tabularnewline
5.0 & 3.6 & 1.4 & 0.2\tabularnewline
5.4 & 3.9 & 1.7 & 0.4\tabularnewline
\bottomrule
\end{longtable}

画像ファイルも貼り付けられる.

\begin{figure}
\centering
\includegraphics[width=0.5\textwidth,height=\textheight]{./2e31456940ff2d1c8e3111c4f9225e4782cfa6d0.jpg}
\caption{Johannes Gutenberg}
\end{figure}

しかし, キャプションを付けたり,
表示位置やサイズを細かく調整したりするためには,
後述するように\textbf{Rプログラムを経由して出力}したほうが良い.

TODO: md 記法で画像貼り付けたときのサイズ統一

\hypertarget{ux30b3ux30e1ux30f3ux30c8ux30a2ux30a6ux30c8}{%
\subsection{コメントアウト}\label{ux30b3ux30e1ux30f3ux30c8ux30a2ux30a6ux30c8}}

HTML 式の \texttt{\textless{}!-\/-\ -\/-\textgreater{}}
でコメントアウトできる.
コメントアウトされた箇所は生成ファイルでもコメントアウトされるのではなく,
そもそも出力されなくなる.

\hypertarget{ux6570ux5f0f}{%
\section{数式}\label{ux6570ux5f0f}}

LaTeX 記法で数式を記述できる. HTML ならば Mathjax
によってレンダリングされる. 数式の記述ルールは少々ややこしい.
これは現在の \texttt{pandoc} の仕様で HTML および LaTeX
の規格で矛盾なく出力するためやむをえない措置である.

\begin{enumerate}
\def\labelenumi{\arabic{enumi}.}
\tightlist
\item
  改行をしない\textbf{行内数式}は \texttt{\$} で囲む, または
  \texttt{\textbackslash{}(}, \texttt{\textbackslash{})} で囲む.
\item
  改行を伴う\textbf{数式ブロック}は \texttt{\$\$} で囲む, または
  \texttt{\textbackslash{}{[}}, \texttt{\textbackslash{}{]}} で囲む.
\item
  \texttt{align}, \texttt{equation} 環境等を使う場合は,
  上記の記号を\textbf{使わず}, 直接 LaTeX コマンド
  \texttt{\textbackslash{}begin\{align\}...} を打ち込む.
\end{enumerate}

\begin{Shaded}
\begin{Highlighting}[]
\NormalTok{\textbackslash{}@ref(eq:binom) は二項分布の確率関数である}
\NormalTok{\textbackslash{}begin\{align\}}
\NormalTok{f(k) \&= \{n \textbackslash{}choose k\} p\^{}\{k\} (1{-}p)\^{}\{n{-}k\} (}\SpecialCharTok{\textbackslash{}\#}\NormalTok{eq:binom)}
\NormalTok{\textbackslash{}end\{align\}}
\end{Highlighting}
\end{Shaded}

その出力は, 以下のようになる.

\eqref{eq:binom} は二項分布の確率関数である

\begin{align}
f(k) &= {n \choose k} p^{k} (1-p)^{n-k} \label{eq:binom}
\end{align}

Bookdown では\textbf{従来の R Markdown でできなかった数式への付番と,
本文中での参照アンカーリンクの自動作成が可能}となっている (詳細は
\ref{crossref} 章で). LaTeX にすでに慣れている読者に注意が必要だが,
Bookdown 特有の制約として, 付番したい場合は
\texttt{\textbackslash{}label\{ID\}} ではなく
\texttt{(\textbackslash{}\#eq:ID)} を使う. また, PDF (LaTeX) と HTML
(Mathjax) の仕様には

\begin{enumerate}
\def\labelenumi{\arabic{enumi}.}
\tightlist
\item
  PDF では \texttt{align} は常に数式が付番され, \texttt{align*}
  等はどうやっても付番されない
\item
  HTML では \texttt{align} でも \texttt{align*}
  であってもラベルを書かなければ付番されず, 書けば付番される.
\end{enumerate}

という違いがある. 両者で同じ表示にこだわるのなら, 付番を取り消す
\texttt{\textbackslash{}notag} を多用することになるだろう.

さらに, bookdown の機能として, LaTeX
の「定理」「定義」「証明」などの環境に対応するものが提供されている
(参考: BKD
\href{https://bookdown.org/yihui/bookdown/markdown-extensions-by-bookdown.html}{Ch.
2.2 Markdown extensions by bookdown}). これらの相互参照も可能である.

例: 以下に補題 \ref{lem:borelcantelli}, 定理 \ref{thm:theorem1} を示す.

\begin{lemma}[ボレル-カンテリの補題]
\protect\hypertarget{lem:borelcantelli}{}{\label{lem:borelcantelli}
\iffalse (ボレル-カンテリの補題) \fi{}
}\({E_1,E_2,\cdots}\)をある確率空間の事象とする.
これらの事象の確率の和が有限であるなら,
それらが無限に多く起こる確率はゼロである. つまり,

\begin{align*}
& \sum_{n=1}^\infty \mathrm{P}(X_n) <\infty \Rightarrow \mathrm{P}\left(\lim_{n\to\infty}\sup X_n\right) = 0,\\
& \lim_{n\to\infty}\sup X_n = \bigcap_{n=1}^\infty\bigcup_{k\leq n}^\infty E_k
\end{align*}
である. \end{lemma}

\begin{proof}
\iffalse{} {証明. } \fi{}証明は読者の課題とする. \end{proof}

\begin{theorem}[無限の猿定理]
\protect\hypertarget{thm:theorem1}{}{\label{thm:theorem1}
\iffalse (無限の猿定理) \fi{}
}猿がほとんど確実にタイプライタの全てのキーを無限回叩くならば,
ほとんど確実にテキストには任意の作品が含まれる. \end{theorem}

\begin{proof}
\iffalse{} {証明. } \fi{}補題 \ref{lem:borelcantelli} より自明.
\end{proof}

\hypertarget{ux30abux30b9ux30bfux30e0ux30d6ux30edux30c3ux30af}{%
\section{カスタムブロック}\label{ux30abux30b9ux30bfux30e0ux30d6ux30edux30c3ux30af}}

数式のセクションの定理ブロックの応用で,
独自のブロックセクションを定義することができる. \texttt{rmdja} では BKD
\href{https://bookdown.org/yihui/bookdown/custom-blocks.html}{Ch. 2.7
Custom blocks} で紹介されている例を予め使えるようにしている. それらは
\texttt{type="..."} で指定できて, 以下の5種類がある.

\begin{itemize}
\tightlist
\item
  \texttt{rmdcaution}
\item
  \texttt{rmdimportant}
\item
  \texttt{rmdnote}
\item
  \texttt{rmdtip}
\item
  \texttt{rmdwarning}
\end{itemize}

である.

\begin{rmdcaution}
技術書によくある注意を喚起するブロック (\texttt{rmdcaution}).
\end{rmdcaution}

\begin{rmdimportant}
技術書によくある注意を喚起するブロック (\texttt{rmdimportant}).
\end{rmdimportant}

\begin{rmdnote}
技術書によくある注意を喚起するブロック (\texttt{rmdcnote}).
\end{rmdnote}

\begin{rmdtip}
技術書によくある注意を喚起するブロック (\texttt{rmdtip}).
\end{rmdtip}

\begin{rmdwarning}
技術書によくある注意を喚起するブロック (\texttt{rmdwarning}).
\end{rmdwarning}

このブロック内では Markdown の基本構文しか使えず,
引用や相互参照などは使えない. これらをブロック内で使いたい場合は
\texttt{block} の代わりに \texttt{block2} と書く. ただしこちらは pandoc
の機能のハックであるため, 将来使えなくなる可能性もある.

\hypertarget{ux811aux6ce8}{%
\section{脚注}\label{ux811aux6ce8}}

脚注はインラインと, 巻末に書く2通りがある.

\begin{Shaded}
\begin{Highlighting}[]
\NormalTok{ここにインラインで脚注\^{}}\CommentTok{[}\OtherTok{脚注の本文}\CommentTok{]}
\end{Highlighting}
\end{Shaded}

ここにインラインで脚注\footnote{脚注の本文}

\begin{Shaded}
\begin{Highlighting}[]
\NormalTok{本文は巻末に書く}\OtherTok{[\^{}example{-}1][\^{}example{-}2]}\NormalTok{.}

\OtherTok{[\^{}example{-}1]: }\NormalTok{脚注の本文その2}
\OtherTok{[\^{}example{-}2]: }\NormalTok{脚注の本文その2}
\end{Highlighting}
\end{Shaded}

本文は巻末に書く\footnote{脚注の本文その2}\footnote{脚注の本文その2}.

ここにインラインで脚注{[}\^{}脚注の本文{]}

インラインで書くほうがシンプルに見えるが,
この記法では間を空けずに連続して脚注を書くことができない.

\begin{Shaded}
\begin{Highlighting}[]
\NormalTok{このように書くと\^{}}\CommentTok{[}\OtherTok{脚注その1}\CommentTok{]}\NormalTok{\^{}}\CommentTok{[}\OtherTok{脚注その2}\CommentTok{]}\NormalTok{上付きとして認識される}
\end{Highlighting}
\end{Shaded}

\hypertarget{ux52d5ux7684ux306aux30b3ux30f3ux30c6ux30f3ux30c4ux306eux4f5cux6210}{%
\chapter{動的なコンテンツの作成}\label{ux52d5ux7684ux306aux30b3ux30f3ux30c6ux30f3ux30c4ux306eux4f5cux6210}}

\hypertarget{ux30d7ux30edux30b0ux30e9ux30e0ux30c1ux30e3ux30f3ux30af}{%
\section{プログラムチャンク}\label{ux30d7ux30edux30b0ux30e9ux30e0ux30c1ux30e3ux30f3ux30af}}

プログラムチャンクは, R Markdown 最大の特徴であり, R のソースコードや,
その実行結果を Markdown に挿入できる. さらには \textbf{R
以外の言語の動作も可能}である. 順番が前後してしまったが,
定理などのカスタムブロックは本来はプログラムを入力するためのチャンクブロックであり,
それを静的なテキストコンテンツの挿入に流用しているだけである.

以降は R で多くのユーザが頻繁に使うパッケージと,
いくつかの技術文書作成に役に立つパッケージをインポートしている前提の説明とする.
なお, \texttt{rmarkdown}, \texttt{bookdown}
はチャンク内で特に読み込む必要がない.

\begin{Shaded}
\begin{Highlighting}[numbers=left,,]
\NormalTok{pkgs \textless{}{-}}\StringTok{ }\KeywordTok{installed.packages}\NormalTok{()}
\ControlFlowTok{for}\NormalTok{ (p }\ControlFlowTok{in} \KeywordTok{c}\NormalTok{(}\StringTok{"tidyverse"}\NormalTok{, }\StringTok{"ggthemes"}\NormalTok{, }\StringTok{"equatiomatic"}\NormalTok{, }\StringTok{"tufte"}\NormalTok{, }\StringTok{"kableExtra"}\NormalTok{)) \{}
  \ControlFlowTok{if}\NormalTok{ (}\OperatorTok{!}\NormalTok{p }\OperatorTok{\%in\%}\StringTok{ }\NormalTok{pkgs) }\KeywordTok{install.packages}\NormalTok{(p)}
\NormalTok{\}}
\ControlFlowTok{if}\NormalTok{ (}\OperatorTok{!}\StringTok{"rmarkdown"} \OperatorTok{\%in\%}\StringTok{ }\NormalTok{pkgs) remotes}\OperatorTok{::}\KeywordTok{install\_github}\NormalTok{(}\StringTok{"rstudio/rmarkdown"}\NormalTok{)}
\ControlFlowTok{if}\NormalTok{ (}\OperatorTok{!}\StringTok{"bookdown"} \OperatorTok{\%in\%}\StringTok{ }\NormalTok{pkgs) remotes}\OperatorTok{::}\KeywordTok{install\_github}\NormalTok{(}\StringTok{"rstudio/bookdown"}\NormalTok{)}
\KeywordTok{require}\NormalTok{(tidyverse)}
\end{Highlighting}
\end{Shaded}

\begin{verbatim}
 要求されたパッケージ tidyverse をロード中です 
\end{verbatim}

\begin{verbatim}
-- Attaching packages --------------------------------------------------------------------------------------------- tidyverse 1.3.0 --
\end{verbatim}

\begin{verbatim}
v ggplot2 3.3.2     v purrr   0.3.4
v tibble  3.0.3     v dplyr   1.0.2
v tidyr   1.1.2     v stringr 1.4.0
v readr   1.3.1     v forcats 0.5.0
\end{verbatim}

\begin{verbatim}
-- Conflicts ------------------------------------------------------------------------------------------------ tidyverse_conflicts() --
x dplyr::filter()     masks stats::filter()
x dplyr::group_rows() masks kableExtra::group_rows()
x dplyr::lag()        masks stats::lag()
\end{verbatim}

\begin{Shaded}
\begin{Highlighting}[numbers=left,,]
\KeywordTok{require}\NormalTok{(ggthemes)}
\end{Highlighting}
\end{Shaded}

\begin{verbatim}
 要求されたパッケージ ggthemes をロード中です 
\end{verbatim}

\begin{Shaded}
\begin{Highlighting}[numbers=left,,]
\KeywordTok{require}\NormalTok{(equatiomatic)}
\end{Highlighting}
\end{Shaded}

\begin{verbatim}
 要求されたパッケージ equatiomatic をロード中です 
\end{verbatim}

\begin{Shaded}
\begin{Highlighting}[numbers=left,,]
\KeywordTok{require}\NormalTok{(kableExtra)}
\end{Highlighting}
\end{Shaded}

このように, ログを掲載することもできる.
これは再現性を重視する際に重宝するが,
一方で単に画像などの主力だけを掲載したい場合もあるだろう. あるいは,
プログラムの解説のため, \textbf{プログラムは掲載するが実行しない},
ということも必要になるかもしれない.
\textbf{プログラムと結果の表示/非表示はどちらも簡単に切り替え可能}である.
そのためには, チャンクオプションを指定する.

\begin{itemize}
\tightlist
\item
  \texttt{echo}: プログラムを掲載するかどうか
\item
  \texttt{message}: プログラム実行結果の標準出力を掲載するかどうか
\item
  \texttt{warning}: プログラム実行結果の警告を掲載するかどうか
\item
  \texttt{error}: プログラム実行結果のエラーを掲載するかどうか
\item
  \texttt{eval}: 文書作成時にプログラムを実行するかどうか
\item
  \texttt{include}: 文書作成時にプログラムを実行し,
  \textbf{かつ掲載しない}かどうか
\item
  \texttt{results}: 出力をいつもの R の出力風にするか (\texttt{markup}),
  隠すか (\texttt{"hide"}), 出力を区切らずまとめるか (\texttt{"hold"}),
  テキストをそのまま出力するか (``\texttt{asis}'').
  最後はソースコードを動的に生成したい場合などに使う (後述).
\end{itemize}

R の論理値は \texttt{TRUE}/\texttt{FALSE} または \texttt{T}/\texttt{F}
と書く.

チャンクごとに個別に設定することも,
デフォルトとして設定することもできる. 前者の場合,

チャンクオプションは \texttt{\{\}} 内部に書く. \texttt{r} は R
で実行するという意味である.

後者の場合, 以下のようなプログラムでデフォルト値を上書きできる.

などと書く. なおこのチャンクは \texttt{eval=F} を設定することで,
実行されることなくプログラムのみ掲載している (ただし,
プログラムのみを掲載するなら, Markdown の機能でも可能である).

\begin{verbatim}
```sh
echo Hello, Bookdown
```
\end{verbatim}

\texttt{\{\}} ブロック内の値にはさらに R プログラムを与えることができる.
この使い方は後の章で解説する.

これらのオプションがあるおかげでプログラムとその結果の再現を説明したい場合はソースコードも表示させたり,
回帰分析やシミュレーションの結果だけを掲載したい時は結果のみ表示したりできる.
これが R Markdown のチャンクの強みである. 例えば Jupyter notebook/lab
などは従来, コードセルと出力セルを自由に隠すことができなかった.

チャンクに使用できる言語は R だけではない. \textbf{つまり Python
なども使用できる}. 以下で対応しているエンジンの一覧を表示できる.

\begin{verbatim}
 [1] "awk"         "bash"        "coffee"      "gawk"        "groovy"     
 [6] "haskell"     "lein"        "mysql"       "node"        "octave"     
[11] "perl"        "psql"        "Rscript"     "ruby"        "sas"        
[16] "scala"       "sed"         "sh"          "stata"       "zsh"        
[21] "highlight"   "Rcpp"        "tikz"        "dot"         "c"          
[26] "cc"          "fortran"     "fortran95"   "asy"         "cat"        
[31] "asis"        "stan"        "block"       "block2"      "js"         
[36] "css"         "sql"         "go"          "python"      "julia"      
[41] "sass"        "scss"        "theorem"     "lemma"       "corollary"  
[46] "proposition" "conjecture"  "definition"  "example"     "exercise"   
[51] "proof"       "remark"      "solution"   
\end{verbatim}

また, 新たにプログラムを追加することもできる. 詳細は RDG
\href{https://bookdown.org/yihui/rmarkdown/language-engines.html}{Ch.
2.7 Other language engines} を参考に.

TODO: 他の言語のプログラムを実行する際の注意点

\hypertarget{ux30d7ux30edux30b0ux30e9ux30e0ux3067ux6570ux5f0fux3092ux751fux6210ux3059ux308b}{%
\section{プログラムで数式を生成する}\label{ux30d7ux30edux30b0ux30e9ux30e0ux3067ux6570ux5f0fux3092ux751fux6210ux3059ux308b}}

プログラムチャンクは, 単にプログラムの計算結果を埋め込むだけでなく,
静的なコンテンツを臨機応変に変更して出力させたり,
あるいは手作業でやるには煩雑な加工処理を挟んでから表示させるのに役に立つ.

R
のプログラムと組み合わせることで\textbf{回帰分析の結果の数値をコピペすることなく数式で表示することができる}.
そのためには
\href{https://github.com/datalorax/equatiomatic}{\texttt{equatiomatic}}
パッケージの \texttt{extract\_eq()} を使う.

まずは, 回帰係数を記号で表現するタイプ. LaTeX
数式をそのまま出力するため, チャンクオプションに \texttt{results="asis"}
を付ける必要があることに注意する.

\[
\begin{aligned}
Sepal.Length &= \alpha + \beta_{1}(Sepal.Width) + \beta_{2}(Petal.Length) + \beta_{3}(Petal.Width)\ + \\
&\quad \beta_{4}(Species_{versicolor}) + \beta_{5}(Species_{virginica}) + \epsilon
\end{aligned}
\]

さらに \texttt{use\_coef\ =\ T} で係数を推定結果の数値に置き換えた.

\[
\begin{aligned}
Sepal.Length &= 2.17 + 0.5(Sepal.Width) + 0.83(Petal.Length) - 0.32(Petal.Width)\ - \\
&\quad 0.72(Species_{versicolor}) - 1.02(Species_{virginica}) + \epsilon
\end{aligned}
\]

\texttt{equatiomatic} パッケージは現時点では \texttt{lm} \texttt{glm}
に対応しており, \texttt{lmer} への対応も進めているようだ.

TODO: この書き方だと PDF で付番できない

\hypertarget{ux30d7ux30edux30b0ux30e9ux30e0ux3092ux4f7fux3063ux305fux56f3ux306eux633fux5165}{%
\section{プログラムを使った図の挿入}\label{ux30d7ux30edux30b0ux30e9ux30e0ux3092ux4f7fux3063ux305fux56f3ux306eux633fux5165}}

既に Markdown 記法による図表の挿入方法を紹介したが,
プログラムチャンクを介して画像を読み込み表示させることもできる. まずは,
R のプログラムで既存の画像ファイルを表示させる方法.

\begin{figure}

{\centering \includegraphics[width=0.5\linewidth,height=0.5\textheight,keepaspectratio]{img//Johannes_Gutenberg} 

}

\caption{Johannes Gutenberg}\label{fig:includegraphic-example}
\end{figure}

もちろんのこと既存の画像だけでなく,
データを読み込んでヒストグラムや散布図などを描いた結果を画像として掲載することもできる.

技術文書や学術論文では, 画像の上か下に「図1:
XXXXX」のような\textbf{キャプション}を付けることが多い.
紙の書籍では絵本のように本文と図の順序を厳密に守るより,
余白を作らないよう図の掲載位置を調整する必要があるからだ.

プログラムチャンクにはこのキャプションを入力するオプション
\texttt{fig.cap} があるため, \textbf{\texttt{plot()}
側でタイトルを付けないほうが良い}. 例えば \texttt{ggplot2}
パッケージの関数を使い以下のようなチャンクを書く\footnote{なお,
  Rユーザーならば標準グラフィック関数である \texttt{plot()}
  関数をご存知だろうが,
  本稿では\textbf{標準グラフィック関数の使用を推奨しない}.
  標準グラフィック関数のデバイスはもともと日本語フォントを想定しておらず,
  OSごとに使用できるフォントも異なるためで,
  品質維持のためには使用させない方針とした.
  工夫すれば標準グラフィック関数でも日本語を適切に出力できるが,
  \texttt{ggplot2} を使用したほうが簡単であることが多いため,
  標準グラフィック関数の解説書を作る以外では使うべきでない.}.

\begin{Shaded}
\begin{Highlighting}[numbers=left,,]
\NormalTok{txt \textless{}{-}}\StringTok{ \textquotesingle{}\textasciigrave{}\textasciigrave{}\textasciigrave{}\{r plot{-}sample, echo=T, fig.cap="\textasciigrave{}ggplot2\textasciigrave{} によるグラフ"\}}
\StringTok{data("diamonds")}
\StringTok{diamonds \textless{}{-} diamonds[sample(1:NROW(diamonds), size =), ]}
\StringTok{ggplot(diamonds, aes(x=carat, y=price, color=clarity)) +}
\StringTok{  geom\_point() +}
\StringTok{  labs( x = "カラット数", y = "価格") + scale\_color\_pander(name = "クラリティ") +}
\StringTok{  theme\_classic(base\_family = "Noto Sans CJK JP") + theme(legend.position = "bottom")\textasciigrave{}\textasciigrave{}\textasciigrave{}\textquotesingle{}}
\KeywordTok{cat}\NormalTok{(txt)}
\end{Highlighting}
\end{Shaded}

\begin{verbatim}
```{r plot-sample, echo=T, fig.cap="`ggplot2` によるグラフ"}
data("diamonds")
diamonds <- diamonds[sample(1:NROW(diamonds), size =), ]
ggplot(diamonds, aes(x=carat, y=price, color=clarity)) +
  geom_point() +
  labs( x = "カラット数", y = "価格") + scale_color_pander(name = "クラリティ") +
  theme_classic(base_family = "Noto Sans CJK JP") + theme(legend.position = "bottom")```
\end{verbatim}

実際の表示は図 \ref{fig:plot-sample} のようになる.

\begin{figure}

{\centering \includegraphics[width=1\linewidth,height=1\textheight,keepaspectratio]{main_files/figure-latex/plot-sample-1} 

}

\caption{`ggplot2` によるグラフ}\label{fig:plot-sample}
\end{figure}

\texttt{ggplot2} 以外のパッケージや言語, たとえば \texttt{tikz} や
\texttt{asymptote}, DOT言語も使用できる. これらは \ref{advanced-graph}
章で紹介する.

\hypertarget{todo-ux56f3ux306eux30ecux30a4ux30a2ux30a6ux30c8ux8a2dux5b9a}{%
\section{TODO:
図のレイアウト設定}\label{todo-ux56f3ux306eux30ecux30a4ux30a2ux30a6ux30c8ux8a2dux5b9a}}

PDF ならばフロート設定のため, 図が離れた位置に配置されることがある.
そのため, 「図 \ref{fig:plot-sample}」
のような相互参照を使うと良いだろう. フロートを使うかどうかは,
後のセクションで解説する TODO

Rのグラフィックデバイスを使っている限り,
通常のRのコンソールと同じコードをチャンク内に書くだけで表示できる.

R のグラフィックデバイスではないとは, RGL や \texttt{plotly}
など外部ライブラリに頼ったグラフ作成ツールのことである.
判断できない人は, RStudio 上で実行して, ``Plots'' ペーンに表示されたら R
のグラフィックデバイス, ``Viewer'' ペーンに表示されたらそうでない,
で覚えていただきたい. 後者を表示する方法は \ref{webapp} 章で後述する. R
をこれまで使ったことがなく, それすらも何を言っているのか分からない,
という場合は \texttt{ggplot2} を使ってもらう.

最後の \texttt{fig.cap=""} がキャプションである. ただし,
どうも日本語キャプションを書いたあとに他のチャンクオプションを指定するとエラーになるようだ.
よって \textbf{\texttt{fig.cap=} はオプションの末尾に書くべきである}.
また, \texttt{fig.cap=""}
に数式や一部の特殊なテキストを直接入力することができない.
この問題は相互参照について解説するセクション \ref{crossref}
で詳細を述べる.

\texttt{fig.cap} 以外のオプションはおそらく頻繁には変えないため,
冒頭でまとめて設定したほうが楽だろう.

\begin{Shaded}
\begin{Highlighting}[numbers=left,,]
\NormalTok{knitr}\OperatorTok{::}\NormalTok{opts\_chunk}\OperatorTok{$}\KeywordTok{set}\NormalTok{(}
  \DataTypeTok{fig.align =} \StringTok{"center"}\NormalTok{,}
  \DataTypeTok{fig.width =} \FloatTok{6.5}\NormalTok{,}
  \DataTypeTok{fig.height =} \FloatTok{4.5}\NormalTok{,}
  \DataTypeTok{out.width =} \StringTok{"100\%"}\NormalTok{,}
  \DataTypeTok{out.height =} \StringTok{"100\%"}
\NormalTok{)}
\end{Highlighting}
\end{Shaded}

なお, これらは \texttt{rmdja} でのデフォルト値であるため,
実際にこの値をあえて記述する必要はない.

ここで, \texttt{fig.width} と \texttt{out.width} の違いも述べておく.
\texttt{out.width}/\texttt{out.height} は表示する画像サイズの違いで,
\texttt{fig.width}/\texttt{fig.height}
はプログラムが出力した画像の保存サイズである. よって \texttt{ggplot2}
などを使わず画像ファイルを貼り付けるだけの場合は \texttt{fig.*}
は意味をなさない.

\hypertarget{r-ux30d7ux30edux30b0ux30e9ux30e0ux3092ux4f7fux3063ux305fux8868ux306eux88c5ux98fe}{%
\section{R
プログラムを使った表の装飾}\label{r-ux30d7ux30edux30b0ux30e9ux30e0ux3092ux4f7fux3063ux305fux8868ux306eux88c5ux98fe}}

Markdown 記法を使った表記は既に紹介した.
しかしこれは表の数値を全て手動で書かなければならない. もちろんこれも R
内のデータを手書きなどせずとも表示できるし,
テーブルのデザインもある程度自由に設定できる.

R Markdown のデフォルトでは R
のコンソールと同様にテキストとして出力されるが, bookdown
では異なるデザインで表示されている. これは \texttt{knitr},
\texttt{kableExtra}
パッケージなどで事後処理をかけることで見やすいデザインの表に変換しているからである.

この方法はシンプルで使いやすいが, R
はテーブル状のデータ処理に長けているため,
手動で数値を書くよりも簡単な方法がある.

\begin{Shaded}
\begin{Highlighting}[numbers=left,,]
\KeywordTok{data}\NormalTok{(iris)}
\KeywordTok{kable}\NormalTok{(}
  \KeywordTok{head}\NormalTok{(iris, }\DataTypeTok{n =} \DecValTok{10}\NormalTok{),}
  \DataTypeTok{caption =} \StringTok{"\textasciigrave{}knitr::kable()\textasciigrave{} で出力された (PDFではあまりかっこよくない) 表"}
\NormalTok{)}
\end{Highlighting}
\end{Shaded}

\begin{table}

\caption{\label{tab:display-dataframe-kable}`knitr::kable()` で出力された (PDFではあまりかっこよくない) 表}
\centering
\begin{tabular}[t]{r|r|r|r|l}
\hline
Sepal.Length & Sepal.Width & Petal.Length & Petal.Width & Species\\
\hline
5.1 & 3.5 & 1.4 & 0.2 & setosa\\
\hline
4.9 & 3.0 & 1.4 & 0.2 & setosa\\
\hline
4.7 & 3.2 & 1.3 & 0.2 & setosa\\
\hline
4.6 & 3.1 & 1.5 & 0.2 & setosa\\
\hline
5.0 & 3.6 & 1.4 & 0.2 & setosa\\
\hline
5.4 & 3.9 & 1.7 & 0.4 & setosa\\
\hline
4.6 & 3.4 & 1.4 & 0.3 & setosa\\
\hline
5.0 & 3.4 & 1.5 & 0.2 & setosa\\
\hline
4.4 & 2.9 & 1.4 & 0.2 & setosa\\
\hline
4.9 & 3.1 & 1.5 & 0.1 & setosa\\
\hline
\end{tabular}
\end{table}

こちらは関数内にキャプションを書く必要があり,
チャンクオプションに指定する方法はない
(表\ref{tab:display-dataframe-kable}).

\begin{Shaded}
\begin{Highlighting}[numbers=left,,]
\KeywordTok{data}\NormalTok{(iris)}
\KeywordTok{kable}\NormalTok{(}
  \KeywordTok{head}\NormalTok{(iris, }\DataTypeTok{n =} \DecValTok{10}\NormalTok{),}
  \DataTypeTok{booktabs =}\NormalTok{ T,}
  \DataTypeTok{caption =} \StringTok{"奇数行を強調し, PDF では booktabs を利用"}
\NormalTok{) }\OperatorTok{\%\textgreater{}\%}\StringTok{ }\KeywordTok{row\_spec}\NormalTok{(}\KeywordTok{seq}\NormalTok{(}\DecValTok{1}\NormalTok{, }\DecValTok{10}\NormalTok{, }\DataTypeTok{by =} \DecValTok{2}\NormalTok{), }\DataTypeTok{background =} \StringTok{"gray"}\NormalTok{)}
\end{Highlighting}
\end{Shaded}

\begin{table}

\caption{\label{tab:display-dataframe-kable-2}奇数行を強調し, PDF では booktabs を利用}
\centering
\begin{tabular}[t]{rrrrl}
\toprule
Sepal.Length & Sepal.Width & Petal.Length & Petal.Width & Species\\
\midrule
\cellcolor{gray}{5.1} & \cellcolor{gray}{3.5} & \cellcolor{gray}{1.4} & \cellcolor{gray}{0.2} & \cellcolor{gray}{setosa}\\
4.9 & 3.0 & 1.4 & 0.2 & setosa\\
\cellcolor{gray}{4.7} & \cellcolor{gray}{3.2} & \cellcolor{gray}{1.3} & \cellcolor{gray}{0.2} & \cellcolor{gray}{setosa}\\
4.6 & 3.1 & 1.5 & 0.2 & setosa\\
\cellcolor{gray}{5.0} & \cellcolor{gray}{3.6} & \cellcolor{gray}{1.4} & \cellcolor{gray}{0.2} & \cellcolor{gray}{setosa}\\
\addlinespace
5.4 & 3.9 & 1.7 & 0.4 & setosa\\
\cellcolor{gray}{4.6} & \cellcolor{gray}{3.4} & \cellcolor{gray}{1.4} & \cellcolor{gray}{0.3} & \cellcolor{gray}{setosa}\\
5.0 & 3.4 & 1.5 & 0.2 & setosa\\
\cellcolor{gray}{4.4} & \cellcolor{gray}{2.9} & \cellcolor{gray}{1.4} & \cellcolor{gray}{0.2} & \cellcolor{gray}{setosa}\\
4.9 & 3.1 & 1.5 & 0.1 & setosa\\
\bottomrule
\end{tabular}
\end{table}

さらに, RCB の
\href{https://bookdown.org/yihui/rmarkdown-cookbook/kable.html}{Ch. 10.1
The function \texttt{knitr::kable()}}
ではその他いろいろな書式設定を紹介している.

\hypertarget{ux76f8ux4e92ux53c2ux7167ux3068ux5f15ux7528}{%
\chapter{相互参照と引用}\label{ux76f8ux4e92ux53c2ux7167ux3068ux5f15ux7528}}

\hypertarget{crossref}{%
\section{相互参照}\label{crossref}}

\hypertarget{ux56f3ux8868ux3084ux5f0fux3078ux306eux30a2ux30f3ux30abux30fcux30eaux30f3ux30af}{%
\subsection{図表や式へのアンカーリンク}\label{ux56f3ux8868ux3084ux5f0fux3078ux306eux30a2ux30f3ux30abux30fcux30eaux30f3ux30af}}

図, 表, 式などに番号を自動で割り当て, さらにハイパーリンクを付加できる.
\texttt{\textbackslash{}@ref(ID)} を使う. 現状では \texttt{refstyle} や
\texttt{prettyref} のように接頭語を自動で付けてくれないが,
そのうちなんとかなるかもしれない.

bookdown の相互参照は, LaTeX の \texttt{prettyref.sty} のように,
\texttt{接頭語:参照ID} という記法になる. 参照IDは通常,
チャンクIDと同じである. 既に紹介したように, 数式参照の接頭語は
\texttt{eq} で, 定理は \texttt{thm} である. 図表は \texttt{fig},
\texttt{tab}. その他の接頭語は BKD
\href{https://bookdown.org/yihui/bookdown/markdown-extensions-by-bookdown.html\#equations}{Ch.
2.2 Markdown extensions by bookdown} を参考に.

\hypertarget{ux8868ux3078ux306eux76f8ux4e92ux53c2ux7167}{%
\subsection{表への相互参照}\label{ux8868ux3078ux306eux76f8ux4e92ux53c2ux7167}}

Markdown 記法で表を書く場合, 以下のように \texttt{Table:}
の直後にラベルを記入する (表 \ref{tab:tab-md}).

\begin{Shaded}
\begin{Highlighting}[]
\AnnotationTok{Table:}\CommentTok{ (\textbackslash{}\#tab:tab{-}md) Markdown 記法の表}

\NormalTok{ Sepal.Length   Sepal.Width   Petal.Length   Petal.Width}
\NormalTok{{-}{-}{-}{-}{-}{-}{-}{-}{-}{-}{-}{-}{-}  {-}{-}{-}{-}{-}{-}{-}{-}{-}{-}{-}{-}  {-}{-}{-}{-}{-}{-}{-}{-}{-}{-}{-}{-}{-}  {-}{-}{-}{-}{-}{-}{-}{-}{-}{-}{-}{-}}
\InformationTok{          5.1           3.5            1.4           0.2}
\InformationTok{          4.9           3.0            1.4           0.2}
\InformationTok{          4.7           3.2            1.3           0.2}
\InformationTok{          4.6           3.1            1.5           0.2}
\InformationTok{          5.0           3.6            1.4           0.2}
\InformationTok{          5.4           3.9            1.7           0.4}
\end{Highlighting}
\end{Shaded}

\begin{longtable}[]{@{}rrrr@{}}
\caption{\label{tab:tab-md} Markdown 記法の表}\tabularnewline
\toprule
Sepal.Length & Sepal.Width & Petal.Length & Petal.Width\tabularnewline
\midrule
\endfirsthead
\toprule
Sepal.Length & Sepal.Width & Petal.Length & Petal.Width\tabularnewline
\midrule
\endhead
5.1 & 3.5 & 1.4 & 0.2\tabularnewline
4.9 & 3.0 & 1.4 & 0.2\tabularnewline
4.7 & 3.2 & 1.3 & 0.2\tabularnewline
4.6 & 3.1 & 1.5 & 0.2\tabularnewline
5.0 & 3.6 & 1.4 & 0.2\tabularnewline
5.4 & 3.9 & 1.7 & 0.4\tabularnewline
\bottomrule
\end{longtable}

\hypertarget{ux7ae0ux3078ux306eux76f8ux4e92ux53c2ux7167}{%
\subsection{章への相互参照}\label{ux7ae0ux3078ux306eux76f8ux4e92ux53c2ux7167}}

章見出しへの相互参照も可能である. これはPandocの機能を利用しているため,
接頭辞は不要である.
Pandocの仕様により欧文であればタイトルがそのまま参照IDとなるが,
非欧文の文字に対して適用されないため,
参照したい章の見出しにの後にスペースを入れて \texttt{\{\#参照ID\}}
と書く必要がある.

\hypertarget{ux7279ux6b8aux306aux76f8ux4e92ux53c2ux7167}{%
\subsection{特殊な相互参照}\label{ux7279ux6b8aux306aux76f8ux4e92ux53c2ux7167}}

チャンクオプションの \texttt{fig.cap} などに TeX
数式を書いても正しく表示できない. そのような場合は \texttt{ref}
参照を使う.
\texttt{(ref:figcap1)\ \textbackslash{}coloremoji\{🌸\}\ \$\textbackslash{}sum\ \textbackslash{}oint\ \textbackslash{}mathfrak\{A\}\ \textbackslash{}mathscr\{B\}\ \textbackslash{}mathbb\{C\}\$\ \textbackslash{}coloremoji\{🌸\}}
と書くと, 図 \ref{fig:caption} のキャプションにも特殊な記号が使える.




なお, 複数指定する場合は連続させず, 改行で1行空けて宣言する必要がある.

\begin{figure}

{\centering \includegraphics[width=0.3\linewidth,height=1\textheight,keepaspectratio]{main_files/figure-latex/caption-1} 

}

\caption{🌸\coloremoji{🌸}
\(\sum \oint \mathfrak{A} \mathscr{B} \mathbb{C}\) \coloremoji{🌸}🌸}\label{fig:caption}
\end{figure}

この参照は\textbf{一度しか使えない}.

PDF での表示では, 図 \ref{fig:caption}
のキャプションの外側が文字化けしていることだろう.
これは絵文字出力に関する問題で, 別のセクションで解説する.

これはかなり強力で,

\begin{enumerate}
\def\labelenumi{\arabic{enumi}.}
\tightlist
\item
  定義される前の行にも適用される
\item
  チャンクオプションだけでなく出力結果にも適用される
\end{enumerate}

という仕様である.

TODO: 自己言及的な文章は書かないならこれくらいの認識でいいだろうが,
より正確な話はどうするか

\hypertarget{bibliography}{%
\section{文献引用}\label{bibliography}}

YAMLフロントマターの \texttt{biblography:} に文献管理ファイル
(\texttt{.bib}, \texttt{.json} 等) を指定することで,
ファイルに含まれる文献への参照が可能になる. \texttt{@引用ID}
で本文に引用を与えられ, 文書に引用した文献の一覧が自動で生成される.
また, \texttt{citr} パッケージにより, RStudio Addins
に文献に対応する引用IDを取り出して挿入する機能が追加される.



\begin{figure}

{\centering \includegraphics[width=1\linewidth,height=0.7\textheight,keepaspectratio]{img//citr} 

}

\caption{\texttt{citr} パッケージの例}\label{fig:citr-image}
\end{figure}

\texttt{BibTeX}, \texttt{BibLaTeX}, または \texttt{pandoc-citeproc}
で処理される. これは \texttt{citation-package} または
\texttt{\_output.yml} の \texttt{citation\_package} に依存する. HTML
は基本的に \texttt{pandoc-citeproc} を使用した出力であり,
影響が大きいのは PDF である.

\begin{itemize}
\tightlist
\item
  \texttt{default}: \texttt{pandoc-citeproc}
  を使用して参考文献を出力する.
\item
  \texttt{biblatex}: \texttt{biblatex} を使用する.
\item
  \texttt{natbib}: \texttt{bibtex} を使用し, 本文中の参照には
  \texttt{natbib.sty} を使う.

  \begin{itemize}
  \tightlist
  \item
    \texttt{natbiboptions:} で \texttt{number}, \texttt{authoryear}
    などの \texttt{natbib.sty} のオプションを指定できる.
  \end{itemize}
\end{itemize}

そのスタイルファイル (\texttt{.bst}, \texttt{.csl}) はそれぞれ
\texttt{biblio-style:}/\texttt{csl:} で指定できる. また,
\texttt{biblatex} の場合はここにオプションを指定することになる.
しかし国内論文雑誌などで規定された, 日本語文献に対応した \texttt{.bst}
ファイルを使って参考文献を記載したい場合には (u)\pBibTeX
が必要となるはずである.

これはR Markdownで日本語技術文書を作る際の特にアレな障害の1つである. PDF
は BibTeX があるが, HTML では使えない. pandoc の出力に頼るしかない.
pandoc は CSL という, Word
でも使われている参考文献リストのフォーマットに対応しているが,
機能が少ないため日本語文献の引用スタイルが崩れてしまう.

日本語対応 \texttt{.bst}
ファイルを使いたい場合は少しトリッキーな操作が必要になる.
\texttt{rmarkdown} は \texttt{tinytex}
というパッケージでインストールされたスタンドアローンな処理系で PDF
を生成している. 冒頭のチャンクで
\texttt{options(tinytex.latexmk.emulation\ =\ F)} を指定することで,
自分のマシンにインストールされている普段使っている処理系に処理させることができる.
さらに \texttt{rmdja} では \texttt{natbib}
を指定した場合に自動でカレントディレクトリに \texttt{.latexmkrc}
をコピーするようにしている.
しかしログが残らないなどデバッグしづらいところがあるため,
このやり方はやや使いづらい.

TODO: この操作なしに (u)\pBibTeX を使うには, たぶん \texttt{tinytex} と
\texttt{rmarkdown} 両方の修正が必要.
そこまで複雑ではないと思うのでそのうち修正してみたい.

なお, 普段文献管理ソフトを使っていないが,
数本程度の文献を引用リストに載せたい利用者は, \texttt{biblatex}
の構文を利用して書くのがよいかもしれない. 例えばここに書いてあるように.
その場合, デフォルトでは本文の引用は {[}1{]}, {[}2{]}
のような番号形式となる. \texttt{biblio-style:\ authoryear} とすることで,
\texttt{natbib} のような 「著者 (出版年)」 スタイルとなる.

\url{https://teastat.blogspot.com/2019/01/bookdown.html}

\hypertarget{ux7c21ux5358ux306aux30ecux30a4ux30a2ux30a6ux30c8ux5909ux66f4}{%
\chapter{簡単なレイアウト変更}\label{ux7c21ux5358ux306aux30ecux30a4ux30a2ux30a6ux30c8ux5909ux66f4}}

\hypertarget{html}{%
\section{HTML}\label{html}}

\hypertarget{ux30d5ux30a9ux30f3ux30c8ux5909ux66f4}{%
\subsection{フォント変更}\label{ux30d5ux30a9ux30f3ux30c8ux5909ux66f4}}

HTML は文字通りHTMLで出力しているため, CSS
の使い方次第でいくらでもデザインを変えることができる.

\hypertarget{pdf}{%
\section{PDF}\label{pdf}}

\hypertarget{ux30d5ux30a9ux30f3ux30c8ux5909ux66f4-1}{%
\subsection{フォント変更}\label{ux30d5ux30a9ux30f3ux30c8ux5909ux66f4-1}}

PDF を生成する場合, ver 0.3 時点ではデフォルトのフォントを OS
に応じて変えている.
もし変更したい場合はYAMLフロントマターの以下の項目を変更する

\begin{itemize}
\tightlist
\item
  \texttt{mainfont}: 欧文セリフフォント
\item
  \texttt{sansfont}: 欧文サンセリフフォント
\item
  \texttt{monofont}: 等幅フォント (コードの表示などに使用)
\item
  \texttt{jfontpreset}: 和文フォントのプリセット
\item
  \texttt{jmainfont}: 和文メインフォント (一般に明朝体を指定)
\item
  \texttt{jsansfont}: 和文セリフフォント (一般にゴシック体を指定)
\item
  \texttt{jmonofont}: 和文等幅フォント (コードの表示などに使用)
\end{itemize}

\texttt{jfontpreset} は \texttt{zxjafont} または \texttt{luatex-ja}
によるプリセットで, 3種類のフォントを一括指定できる.
個別指定したフォントはこれを上書きする.
特にこだわりがないなら一括指定で良いが, ソースコードを多く掲載する場合は
\texttt{M+} や \texttt{Ricty} などのフォントを用意すると良いだろう.

さらに, それぞれの項目には \texttt{options}
と接尾辞のついた項目が用意されている.
フォントの相対サイズが合わず不格好な場合は

\begin{Shaded}
\begin{Highlighting}[]
\FunctionTok{mainfont}\KeywordTok{:}\AttributeTok{ Palatinno}
\FunctionTok{mainfontoptions}\KeywordTok{:}
\AttributeTok{  }\KeywordTok{{-}}\AttributeTok{ Scale=0.9}
\end{Highlighting}
\end{Shaded}

などと書いて調整できる.

インラインのフォント変更は TODO

\hypertarget{ux6587ux66f8ux30afux30e9ux30b9ux306eux5909ux66f4}{%
\subsection{文書クラスの変更}\label{ux6587ux66f8ux30afux30e9ux30b9ux306eux5909ux66f4}}

HTMLは利用者側が見え方をある程度カスタマイズできる. かつて存在した
Evernote Clearly やカスタム CSS を使って.
そのぶんPDFは作成者側がよりレイアウトに注意を払うことになるだろう.
本稿では文章の区切りを章立てにしている. しかし PDF
数十ページしかない文書を大きな文字サイズの見出しで区切るのは少しものものしい感じがする.
YAML フロントマターを変更すれば, トップレベルの見出しを変更できる.

まず, 今回は文書ということで書籍の組版をデフォルト設定にしている.
もう少し小規模な文書ならば,
\textbf{レポート}や\textbf{論文記事形式}のほうが良いかもしれない.
例えば, 以下のように指定する.

\begin{Shaded}
\begin{Highlighting}[]
\FunctionTok{documentclass}\KeywordTok{:}\AttributeTok{ bxjsreport}
\end{Highlighting}
\end{Shaded}

\texttt{documentclass} には LaTeX の文書クラスファイル (\texttt{.cls})
ならなんでも与えることができるが, \XeLaTeX または
\LuaLaTeX で日本語文書を作成することを想定しているため, BXjscls
シリーズのクラスから選ぶことを推奨する\footnote{\url{https://www.ctan.org/pkg/bxjscls}.
  但し, スライド用クラスである \texttt{bxjsslide}
  の使用は想定していない.}. よって, 以下3種類の中から選ぶとよい.
デフォルトは \texttt{bxjsbook} なので, これは明示的に指定する必要はない.

\begin{itemize}
\tightlist
\item
  \texttt{bxjsbook}
\item
  \texttt{bxjsreport}
\item
  \texttt{bxjsarticle}
\end{itemize}

このうち, \texttt{bxjsbook} がデフォルト設定となっている.

文書クラスとは別に, 文書

その他, \texttt{\_output.yml} や \texttt{\_bookdown.yml}
のコメントを参考に.

\hypertarget{rmdja-ux306bux3088ux308bux6587ux66f8ux4f5cux6210ux652fux63f4ux6a5fux80fd}{%
\chapter{\texorpdfstring{\texttt{rmdja}
による文書作成支援機能}{rmdja による文書作成支援機能}}\label{rmdja-ux306bux3088ux308bux6587ux66f8ux4f5cux6210ux652fux63f4ux6a5fux80fd}}

\hypertarget{ux30afux30eaux30a8ux30a4ux30c6ux30a3ux30d6ux30b3ux30e2ux30f3ux30baux306eux8868ux8a18}{%
\subsection{クリエイティブ・コモンズの表記}\label{ux30afux30eaux30a8ux30a4ux30c6ux30a3ux30d6ux30b3ux30e2ux30f3ux30baux306eux8868ux8a18}}

Web公開する文書ならばクリエイティブ・コモンズの表記をつけたいところだ.
公式サイトで毎回発行するのは面倒なので表示する関数を用意にした.
ハイパーリンクも付けるようにしている. チャンクでは
\texttt{results="asis"} オプションが必要になる. また, 通常は
\texttt{echo=F} を設定すべきだろう. 冒頭の表記もこれで作成している.
もちろんそれぞれの媒体に対応している.

文言の生成は未対応

\hypertarget{ux30ebux30d3ux8868ux8a18}{%
\subsection{ルビ表記}\label{ux30ebux30d3ux8868ux8a18}}

ルビはおそらくCJK言語など一部の言語でしか使われていない
(アラビア語とかヘブライ語とかの補助記号は詳しく知らないが多分グリフとしてサポートされてるっぽいので無視)
ため, ルビ表記も R Markdown ではサポートされていない.
そこで簡単にルビを表示できる関数 \texttt{rmdja::ruby()} を用意した.
インライン実行で使う. PDF での配置は \texttt{pxrubrica.sty}
を利用したグループルビである. よって, 1字ごとに配置 (モノルビ)
にしたいとか, 突出指定とか,
細かいことはHTMLタグやCSSやLaTeXコマンドを自分で書く. 妥協案として,
1字ごとに呼び出す手もある.

グループルビの例: とある科学の\ruby[g]{超電磁砲}{レールガン},
\ruby[g]{皇帝}{カイザー}ラインハルト,
\ruby[g]{柊館}{シュテッヒパルムシュロス},
\ruby[g]{黒色槍騎兵}{シュワルツ・ランツェンレイター},
\ruby[g]{喜連瓜破}{きれうりわり}, , \ruby[g]{MEXICO}{メキシコ}

分割して出力した例:
\ruby[g]{喜}{き}\ruby[g]{連}{れ}\ruby[g]{瓜}{うり}\ruby[g]{破}{わり},
\ruby[g]{黒色}{シュワルツ}\ruby[g]{槍騎兵}{ランツェンレイター} ,

TODO: それ以外にも便利機能を少しづつ増やしていく予定

\hypertarget{part-ux5fdcux7528ux7de8}{%
\part{応用編}\label{part-ux5fdcux7528ux7de8}}

\hypertarget{ux3053ux306eux30d1ux30fcux30c8ux306bux3064ux3044ux3066}{%
\chapter*{このパートについて}\label{ux3053ux306eux30d1ux30fcux30c8ux306bux3064ux3044ux3066}}
\addcontentsline{toc}{chapter}{このパートについて}

このパートでは, ここまでで紹介した基本機能の応用で, さまざまな R
パッケージやその他の外部プログラムの出力を埋め込む方法を紹介する

\hypertarget{advanced-graph}{%
\chapter{様々なグラフィックプログラムの埋め込み}\label{advanced-graph}}

\hypertarget{tikz-ux3092ux4f7fux3046}{%
\section{\texorpdfstring{\texttt{tikz}
を使う}{tikz を使う}}\label{tikz-ux3092ux4f7fux3046}}

\LaTeX で使われる \texttt{tikzdevice} を利用して, 直接 \texttt{tikz}
の記述による画像を埋め込むことができる. チャンクのエンジンを
\texttt{tikz} とすることで使用でき, 相互参照やキャプション,
画像サイズの指定といったチャンクオプションも使える. 図
\ref{fig:tikz-venn} は \texttt{tikz} で生成した図である.
これはHTMLでも表示できる.

\begin{figure}

{\centering \includegraphics[width=1\linewidth,height=0.5\textheight,keepaspectratio]{main_files/figure-latex/tikz-venn-1} 

}

\caption{tikzを利用した図の表示}\label{fig:tikz-venn}
\end{figure}

\hypertarget{asymptote-ux3092ux4f7fux3046}{%
\section{Asymptote を使う}\label{asymptote-ux3092ux4f7fux3046}}

同様に, Asymptote のプログラムを埋め込むこともできる. 私は Asymptote
が分からないので RCB
\href{https://bookdown.org/yihui/rmarkdown-cookbook/eng-asy.html}{Ch.
15.9 Create graphics with Asymptote} と同様のプログラムを書いておく. (図
\ref{fig:asymptote-graph}).

\begin{figure}

{\centering \includegraphics[width=1\linewidth,height=0.5\textheight,keepaspectratio]{main_files/figure-latex/asymptote-graph-1} 

}

\caption{Asymptote による画像}\label{fig:asymptote-graph}
\end{figure}

\hypertarget{todo-ux305dux306eux4ed6ux306eux30d7ux30edux30b0ux30e9ux30e0}{%
\section{TODO
その他のプログラム}\label{todo-ux305dux306eux4ed6ux306eux30d7ux30edux30b0ux30e9ux30e0}}

D3.js なども使える

\hypertarget{todo-ux305dux306eux4ed6ux306e-r-ux30d7ux30edux30b0ux30e9ux30e0}{%
\section{TODO: その他の R
プログラム}\label{todo-ux305dux306eux4ed6ux306e-r-ux30d7ux30edux30b0ux30e9ux30e0}}

なお, DOT言語は \texttt{DiagrammeR}
パッケージを経由して使うこともできる\footnote{\url{https://bookdown.org/yihui/rmarkdown-cookbook/diagrams.html}}(図:
\ref{fig:diagrammer-graph}).
グラフィカルモデルの記述などはこちらのほうが簡単かもしれない.




\begin{Shaded}
\begin{Highlighting}[numbers=left,,]
\NormalTok{DiagrammeR}\OperatorTok{::}\KeywordTok{grViz}\NormalTok{(}\StringTok{"digraph \{}
\StringTok{  graph [layout = dot, rankdir = TB]}
\StringTok{  }
\StringTok{  node [shape = rectangle]        }
\StringTok{  rec1 [label = \textquotesingle{}Step 1. 起床する\textquotesingle{}]}
\StringTok{  rec2 [label = \textquotesingle{}Step 2. コードを書く\textquotesingle{}]}
\StringTok{  rec3 [label =  \textquotesingle{}Step 3. ???\textquotesingle{}]}
\StringTok{  rec4 [label = \textquotesingle{}Step 4. 給料をもらう\textquotesingle{}]}
\StringTok{  }
\StringTok{  \# edge definitions with the node IDs}
\StringTok{  rec1 {-}\textgreater{} rec2 {-}\textgreater{} rec3 {-}\textgreater{} rec4}
\StringTok{  \}"}\NormalTok{,}
  \DataTypeTok{height =} \DecValTok{500}
\NormalTok{)}
\end{Highlighting}
\end{Shaded}

\begin{figure}

{\centering \includegraphics[width=1\linewidth,height=1\textheight,keepaspectratio]{main_files/figure-latex/diagrammer-graph-1} 

}

\caption{\texttt{DiagrammeR} によるグラフィカルモデル (RCB,
Ch. 4.15 より)}\label{fig:diagrammer-graph}
\end{figure}

\hypertarget{ux8868ux306eux63b2ux8f09}{%
\chapter{表の掲載}\label{ux8868ux306eux63b2ux8f09}}

\hypertarget{texhtml-ux3092ux51faux529bux3059ux308bux95a2ux6570}{%
\section{TeX/HTML
を出力する関数}\label{texhtml-ux3092ux51faux529bux3059ux308bux95a2ux6570}}

\texttt{stargazer} や \texttt{xtable} のように, HTML や \LaTeX
形式で表を出力してくれるパッケージがある. これらは
\texttt{results=\textquotesingle{}asis\textquotesingle{}}
のチャンクオプションを指定することで関数の出力するテキストをそのまま埋め込むことができる.
よって, あとは HTMLか\LaTeX
かといった出力形式の違いに気をつければ表示できる.



\begin{Shaded}
\begin{Highlighting}[numbers=left,,]
\KeywordTok{require}\NormalTok{(stargazer)}
\end{Highlighting}
\end{Shaded}

\begin{verbatim}
 要求されたパッケージ stargazer をロード中です 
\end{verbatim}

\begin{verbatim}
Please cite as: 
\end{verbatim}

\begin{verbatim}
 Hlavac, Marek (2018). stargazer: Well-Formatted Regression and Summary Statistics Tables.
\end{verbatim}

\begin{verbatim}
 R package version 5.2.2. https://CRAN.R-project.org/package=stargazer 
\end{verbatim}

\begin{Shaded}
\begin{Highlighting}[numbers=left,,]
\KeywordTok{stargazer}\NormalTok{(mtcars, }\DataTypeTok{type =} \ControlFlowTok{if}\NormalTok{ (knitr}\OperatorTok{::}\KeywordTok{is\_latex\_output}\NormalTok{()) }\StringTok{"latex"} \ControlFlowTok{else} \StringTok{"html"}\NormalTok{, }\DataTypeTok{header =}\NormalTok{ F)}
\end{Highlighting}
\end{Shaded}

\begin{table}[!htbp] \centering 
  \caption{} 
  \label{} 
\begin{tabular}{@{\extracolsep{5pt}}lccccccc} 
\\[-1.8ex]\hline 
\hline \\[-1.8ex] 
Statistic & \multicolumn{1}{c}{N} & \multicolumn{1}{c}{Mean} & \multicolumn{1}{c}{St. Dev.} & \multicolumn{1}{c}{Min} & \multicolumn{1}{c}{Pctl(25)} & \multicolumn{1}{c}{Pctl(75)} & \multicolumn{1}{c}{Max} \\ 
\hline \\[-1.8ex] 
mpg & 32 & 20.091 & 6.027 & 10 & 15.4 & 22.8 & 34 \\ 
cyl & 32 & 6.188 & 1.786 & 4 & 4 & 8 & 8 \\ 
disp & 32 & 230.722 & 123.939 & 71 & 120.8 & 326 & 472 \\ 
hp & 32 & 146.688 & 68.563 & 52 & 96.5 & 180 & 335 \\ 
drat & 32 & 3.597 & 0.535 & 2.760 & 3.080 & 3.920 & 4.930 \\ 
wt & 32 & 3.217 & 0.978 & 1.513 & 2.581 & 3.610 & 5.424 \\ 
qsec & 32 & 17.849 & 1.787 & 14.500 & 16.892 & 18.900 & 22.900 \\ 
vs & 32 & 0.438 & 0.504 & 0 & 0 & 1 & 1 \\ 
am & 32 & 0.406 & 0.499 & 0 & 0 & 1 & 1 \\ 
gear & 32 & 3.688 & 0.738 & 3 & 3 & 4 & 5 \\ 
carb & 32 & 2.812 & 1.615 & 1 & 2 & 4 & 8 \\ 
\hline \\[-1.8ex] 
\end{tabular} 
\end{table}

\hypertarget{todo-gt-ux30d1ux30c3ux30b1ux30fcux30b8}{%
\section{\texorpdfstring{TODO: \texttt{gt}
パッケージ}{TODO: gt パッケージ}}\label{todo-gt-ux30d1ux30c3ux30b1ux30fcux30b8}}

TODO

\hypertarget{webapp}{%
\chapter{TODO Web アプレットの挿入}\label{webapp}}

\hypertarget{todo-plotly}{%
\subsection{TODO: plotly}\label{todo-plotly}}

\hypertarget{todo-shiny}{%
\subsection{TODO: shiny}\label{todo-shiny}}

\hypertarget{python}{%
\chapter{TODO Python スクリプトの埋め込み}\label{python}}

グラフィックデバイスをどうするかの話

TODO

\hypertarget{part-ux88fdux672cux3068ux591aux69d8ux306aux5f62ux5f0fux3078ux306eux5bfeux5fdc}{%
\part{
製本と多様な形式への対応}\label{part-ux88fdux672cux3068ux591aux69d8ux306aux5f62ux5f0fux3078ux306eux5bfeux5fdc}}

\hypertarget{ux88fdux672cux65b9ux6cd5ux306eux8a73ux7d30}{%
\chapter{製本方法の詳細}\label{ux88fdux672cux65b9ux6cd5ux306eux8a73ux7d30}}

冒頭のチュートリアルで行った製本 (ビルド)
の仕組みをもう少し詳しく解説する.

bookdown-demo を念頭に置いた解説. \texttt{rmdja} も基本的に同じ.

\begin{itemize}
\tightlist
\item
  \texttt{index.Rmd}: デフォルトで最初に読み込まれる\texttt{Rmd}
  ファイル (名前を変える機能もあるが,
  現時点では不具合が起こりやすいのでおすすめしない)
\item
  それ以外の \texttt{Rmd} ファイル: 連結して読み込むことが可能
\item
  \texttt{\_output.yml}: マルチメディア展開のための設定. PDF, HTML, EPUB
  それぞれの設定を書く
\item
  \texttt{\_bookdown.yml}: bookdown のレイアウト設定
\item
  その他の設定ファイル: その他製本に必要なもの, 画像ファイル,
  \texttt{.css} ファイル, \texttt{.bib} 等
\end{itemize}

\texttt{\_output.yaml}, \texttt{\_bookdown.yml} は \texttt{index.Rmd}
のヘッダに書くこともできるが, 長くなりすぎるので分割できる.
\texttt{bookdown::render\_book()} 関数は,
ルートディレクトリのこれらを自動で読み込んでくれる.

\hypertarget{ux30d5ux30a1ux30a4ux30ebux69cbux6210}{%
\section{ファイル構成}\label{ux30d5ux30a1ux30a4ux30ebux69cbux6210}}

これらのファイルの中身を解説する.

\hypertarget{output.yml}{%
\subsection{\texorpdfstring{\texttt{\_output.yml}}{\_output.yml}}\label{output.yml}}

本来の YAML の \texttt{output:} 以下の記述をこの \texttt{\_output.yml}
ファイルに書くことができる. \texttt{output:}
を複数書くと\texttt{rmarkdown::render\_site()}
やビルドツールでそれぞれの形式に一括作成してくれる.

\begin{Shaded}
\begin{Highlighting}[]
\FunctionTok{output}\KeywordTok{:}
\AttributeTok{  bookdown:}\FunctionTok{:gitbook}\KeywordTok{:}
\AttributeTok{    }\FunctionTok{lib\_dir}\KeywordTok{:}\AttributeTok{ assets}
\AttributeTok{    }\FunctionTok{split\_by}\KeywordTok{:}\AttributeTok{ section}
\AttributeTok{    }\FunctionTok{config}\KeywordTok{:}
\AttributeTok{      }\FunctionTok{toolbar}\KeywordTok{:}
\AttributeTok{        }\FunctionTok{position}\KeywordTok{:}\AttributeTok{ static}
\AttributeTok{  bookdown:}\FunctionTok{:pdf\_book}\KeywordTok{:}
\AttributeTok{    }\FunctionTok{keep\_tex}\KeywordTok{:}\AttributeTok{ }\CharTok{yes}
\AttributeTok{  bookdown:}\FunctionTok{:html\_book}\KeywordTok{:}
\AttributeTok{    }\FunctionTok{css}\KeywordTok{:}\AttributeTok{ toc.css}
\FunctionTok{documentclass}\KeywordTok{:}\AttributeTok{ book}
\end{Highlighting}
\end{Shaded}

詳しくは BKD
``\href{https://bookdown.org/yihui/bookdown/output-formats.html}{Ch. 3
Output Formats}'' の章を.

\hypertarget{bookdown.yml}{%
\subsection{\texorpdfstring{\texttt{\_bookdown.yml}}{\_bookdown.yml}}\label{bookdown.yml}}

\texttt{\_bookdown.yml} も \texttt{index.Rmd} の YAML ヘッダの
\texttt{bookdown:} 以下に対応する内容を書くことができる. 例えばどの Rmd
ファイルを読み込むかとか, LaTeX のときだけ, HTML
のときだけ読み込むような設定も可能.

\url{https://ill-identified.hatenablog.com/entry/2020/09/05/202403}

詳しくは, BKD
\href{https://bookdown.org/yihui/bookdown/configuration.html}{Ch. 4.4
Configuration}

Build ペーンから文書をビルドするには, \texttt{index.Rmd} のYAML ヘッダに
\texttt{site:\ bookdown::bookdown\_site} を書く必要がある. さらに,
\texttt{index.Rmd}
をプロジェクトディレクトリのルートに置いていない場合は, ツールバーの
\texttt{Build} -\textgreater{} \texttt{Configure\ Build\ Tools...} から
\texttt{index.Rmd} を置いているディレクトリを site
ディレクトリとする設定が必要になる(図 \ref{fig:build-pane2-1},
\ref{fig:build-pane2-2}).

\begin{figure}

{\centering \includegraphics[width=1\linewidth,height=1\textheight,keepaspectratio]{img//build-pane} 

}

\caption{Build ペーンの手動設定}\label{fig:build-pane2-1}
\end{figure}
\begin{figure}

{\centering \includegraphics[width=1\linewidth,height=1\textheight,keepaspectratio]{img//build-pane-build} 

}

\caption{Build ペーンの手動設定}\label{fig:build-pane2-2}
\end{figure}

または,
\texttt{bookdown::render\_book("index.Rmd",\ "rmdja::pdf\_book\_ja")}
などでも実行できるから, コマンドラインからも実行できる. 同時製本は
\texttt{rmarkdown::render\_site()}.

\hypertarget{ux51faux529bux5f62ux5f0fux306bux3088ux308bux8868ux73feux306eux9650ux754c}{%
\chapter{出力形式による表現の限界}\label{ux51faux529bux5f62ux5f0fux306bux3088ux308bux8868ux73feux306eux9650ux754c}}

\hypertarget{htmlux3068pdfux3067ux51e6ux7406ux3092ux5834ux5408ux5206ux3051ux3059ux308b}{%
\section{HTMLとPDFで処理を場合分けする}\label{htmlux3068pdfux3067ux51e6ux7406ux3092ux5834ux5408ux5206ux3051ux3059ux308b}}

出力方法で言えば, HTML と PDF に大別できる. Rmdは HTMLタグも LaTeX
コマンドも受け付けるが, それぞれ HTML と PDF
に変換する際にしか反映できない. よって, 例えば複雑な図表を LaTeX
コマンドでじかに Rmd ファイルに書いてしまった場合, HTML
では表示されない.

紙媒体と電子媒体では表現できることに差がある.
例えば紙はあらゆる環境で同じような見た目になるが,
ハイパーリンクは付けられないし, 一度出版してしまうと修正は容易ではない.
PDF の見た目も読者の環境に依存しにくいが, やはり更新が容易ではない.

\texttt{bookdown}
には既に印刷された本の中身を書き換えるする機能はないが,
出力ごとに内容を変えることで, PDF にのみ更新履歴を表示することはできる.

\texttt{knitr::is\_latex\_output()}, \texttt{knitr::is\_html\_output()}
などは, knit 時にどの媒体への変換処理なのかを判定するのに使える.
\texttt{rmdja::ruby()} もこの機能を利用しているし, 本文中の \LaTeX
のロゴも HTML と PDF で使い分けている.

また, \texttt{\_bookdown.yml} の設定, \texttt{rmd\_files} は,
媒体別に設定することができる.

\begin{Shaded}
\begin{Highlighting}[]
\FunctionTok{rmd\_files}\KeywordTok{:}
\AttributeTok{  }\FunctionTok{html}\KeywordTok{:}
\AttributeTok{    }\KeywordTok{{-}}\AttributeTok{ index.Rmd}
\AttributeTok{    }\KeywordTok{{-}}\AttributeTok{ html{-}only.Rmd}
\AttributeTok{  }\FunctionTok{latex}\KeywordTok{:}
\AttributeTok{    }\KeywordTok{{-}}\AttributeTok{ index.Rmd}
\AttributeTok{    }\KeywordTok{{-}}\AttributeTok{ latex{-}only.Rmd}
\end{Highlighting}
\end{Shaded}

\hypertarget{ux7d75ux6587ux5b57ux306eux51faux529b}{%
\section{絵文字の出力}\label{ux7d75ux6587ux5b57ux306eux51faux529b}}

絵文字をHTMLでもPDFでも出力したい場合,
\texttt{\textbackslash{}coloremoji\{⛄\}} のように絵文字を囲む. ただし,
RStudio
のエディタは一部のマルチバイト文字の表示に対応していないので予期せぬ不具合に注意する.

現在の主要Webブラウザでは, 特に設定せずとも Unicode
絵文字をカラー画像に置き換えて表示できるものが多い. しかし PDF
生成時には明示的にフォントを指定するか,
画像に置き換える記述が必要である. その実現のため \texttt{bxcoloremoji}
という LaTeX パッケージ\footnote{\url{https://github.com/zr-tex8r/BXcoloremoji}}を利用する.
このパッケージは CTAN に登録されていないため,
別途インストールする必要がある.

\hypertarget{ux753bux50cfux306eux4fddux5b58ux5f62ux5f0f}{%
\section{画像の保存形式}\label{ux753bux50cfux306eux4fddux5b58ux5f62ux5f0f}}

技術文書での画像の多くはプロットなど単純な図形なので,
写真などを掲載するのでない限り, PDF で出力する場合はプロット画像も PDF
にするのが望ましい. JPG や PNG などのラスタ画像では拡大すると粗くなるが,
PDF などのベクタ画像ならば拡大しても粗くならず,
かつ単純な図形ならばはファイルサイズも小さく済むことが多い. 一方で HTML
は通常 Webブラウザで閲覧するため, PDF に対応していないことが多い. HTML
でベクタ画像を掲載したい場合は \textbf{SVG 形式} で出力する.

R による SVG への出力は, 従来組み込みの \texttt{SVG()}
で行うことが多かったが, 近年は新たなパッケージが出ている. 有力なのは
\texttt{svglite} と \texttt{rsvg} である.

\url{https://oku.edu.mie-u.ac.jp/~okumura/stat/svg.html}

\texttt{rsvg} のほうが高性能だが, \texttt{knitr} で対応しているのは
\texttt{svglite} なので簡単に使いたいならこちらを推奨する.

\hypertarget{ux30c7ux30d5ux30a9ux30ebux30c8ux306eux4fddux5b58ux5f62ux5f0f}{%
\section{デフォルトの保存形式}\label{ux30c7ux30d5ux30a9ux30ebux30c8ux306eux4fddux5b58ux5f62ux5f0f}}

デフォルトでは, PDF は \texttt{cairo\_pdf}, HTML
では解像度を高めに設定した \texttt{PNG} を使用している. これは,
件数の多い散布図など,
ベクタ形式ではファイルサイズが大きくなりすぎる場合もありうるための判断である.

画像形式を変更したい場合は, チャンクオプションの \texttt{dev} で,
オプションは \texttt{dev.args=list(...)} で変更できる.

\url{https://bookdown.org/yihui/rmarkdown-cookbook/graphical-device.html}

\hypertarget{ux88fdux672cux3057ux305fux6587ux66f8ux3092ux914dux5e03ux3059ux308b}{%
\chapter{製本した文書を配布する}\label{ux88fdux672cux3057ux305fux6587ux66f8ux3092ux914dux5e03ux3059ux308b}}

\hypertarget{wepux30daux30fcux30b8ux306eux30dbux30b9ux30c6ux30a3ux30f3ux30b0}{%
\section{Wepページのホスティング}\label{wepux30daux30fcux30b8ux306eux30dbux30b9ux30c6ux30a3ux30f3ux30b0}}

HTML ファイルは様々な配布方法がある. もちろん自分でサーバを立てても良い.
特に簡単なのは以下の2点である.

\begin{enumerate}
\def\labelenumi{\arabic{enumi}.}
\tightlist
\item
  github pages を利用する
\item
  bookdown.org に投稿する
\end{enumerate}

\begin{enumerate}
\def\labelenumi{(\arabic{enumi})}
\tightlist
\item
  の詳細は github.com
  の\href{https://docs.github.com/ja/github/working-with-github-pages/about-github-pages}{公式ドキュメント}を見るのが一番良いだろう.
\end{enumerate}

既にbookdownで作成した文書を公開している例は多数ある.
例えば既に何度も言及した公式解説サイトはそれじたいが \texttt{bookdown}
で作られているし, ``R for Data Science'' \autocite{wickham2016Data}
\footnote{\url{https://github.com/hadley/r4ds}}は,
内容の良さも含め一見に値する. また, ``Hands-On Data Visualization:
Interactive Storytelling from Spreadsheets to Code''
\autocite{doughertyforthcomingHandsOn}
という本\footnote{ソース: \url{https://github.com/handsondataviz/book}}が来年出るらしい.
そして面白いことにこれは R Google スプレッドシートとか R
以外のWeb上のサービスの利用法を紹介する文書である.

これらはいずれもソースコードまで公開されている.
もちろんここでいうソースコードとは,
本文中のプログラムだけでなく文書を生成する \texttt{Rmd}
ファイルなども含める.

それ以外にも有名無名の多くのドキュメントが公開されているが,
一方で日本語はまだまだ少ない. 内容が豊富で,
かつソースコードまで公開されている例として以下が見つかった.

\begin{itemize}
\tightlist
\item
  『\href{https://shohei-doi.github.io/quant_polisci/}{Rで計量政治学入門}\footnote{ソース:
    \url{https://github.com/shohei-doi/quant_polisci}}』
\item
  『\href{https://mtoyokura.github.io/Ethics-for-A-Level-Japanese/}{AIレベルの倫理学}\footnote{ソース:
    \url{https://github.com/MToyokura/Ethics-for-A-Level-Japanese}}』
\end{itemize}

bookdown の機能や見た目を確認することができる.
さらに以下2つは私が作成したものである.

\begin{itemize}
\tightlist
\item
  『\href{https://gedevan-aleksizde.github.io/Japan.R2019/}{三國志で学ぶデータ分析
  (Japan.R 2019)}』\footnote{ソース:
    \url{https://github.com/Gedevan-Aleksizde/Japan.R2019}} (Japan.R
  2019 の資料)
\item
  『\href{https://gedevan-aleksizde.github.io/20190125_tokyor/}{経済学と反事実分析
  接触篇 Economics and Counterfactual Analysis: A Contact}』\footnote{ソース:
    \url{https://github.com/Gedevan-Aleksizde/20190125_tokyor}} (Tokyo.R
  第83回の資料)
\end{itemize}

特に私の2作品は PDF のレイアウトにも注意を払っているが, 当時は kazutan
氏作の
\href{https://github.com/kazutan/bookdown_ja_template}{\texttt{bookdown\_ja\_template}}
をさらに改良した
\href{https://github.com/kenjimyzk/bookdown_ja_template}{kenjimyzk
氏のテンプレート} を元にワンオフで作成したフォーマットを使用し,
\texttt{rmdja} を使用していないため,
あまりスマートでない書き方が見られる.

また, HTML 形式の文書には PDF
など他のファイル形式のダウンロードリンクを設置することができる. これは
\texttt{\_bookdown.yml} で表示を指定できる.

\hypertarget{todo-ux5165ux7a3fux3059ux308bux306bux306f}{%
\section{TODO:
入稿するには}\label{todo-ux5165ux7a3fux3059ux308bux306bux306f}}

国内の印刷所で PDF 入稿する際のスタンダードは何だろうか?
紙媒体でやったことがないので全くわからない. ver. 0.3
時点での対応を紹介する.

\hypertarget{ux30c8ux30f3ux30dcux306eux8868ux793a}{%
\subsection{トンボの表示}\label{ux30c8ux30f3ux30dcux306eux8868ux793a}}

\texttt{\_output.yml} で

\begin{Shaded}
\begin{Highlighting}[]
\AttributeTok{rmdja:}\FunctionTok{:pdf\_book\_ja}\KeywordTok{:}
\AttributeTok{  tombow:true}
\end{Highlighting}
\end{Shaded}

とするとPDFにトンボ (trimming mark) を表示する. これは
\texttt{gentombow.sty} によるものである.
しかし私はこの出力が適切なのか判断することができない.

\hypertarget{ux30d5ux30a9ux30f3ux30c8ux306eux57cbux3081ux8fbcux307f}{%
\subsection{フォントの埋め込み}\label{ux30d5ux30a9ux30f3ux30c8ux306eux57cbux3081ux8fbcux307f}}

少なくとも PDF ではフォントを埋め込みそこなったり, Type 3
フォントが設定されないようにしている. ただし Python
等を利用して描いたPDFは個別に設定が必要な場合があり, 保証できない.

TODO: \url{https://teastat.blogspot.com/2019/01/bookdown.html}
の記述のうち, まだ対応してないものがある.

\hypertarget{part-ux5fdcux7528}{%
\part{応用}\label{part-ux5fdcux7528}}

\hypertarget{todo-ux6587ux82b8ux4f5cux54c1ux306eux4f5cux6210ux306eux305fux3081ux306b}{%
\chapter{TODO:
文芸作品の作成のために}\label{todo-ux6587ux82b8ux4f5cux54c1ux306eux4f5cux6210ux306eux305fux3081ux306b}}

作家の京極夏彦は自分の作品を1ページごとに切り取っても作品として成立するようなレイアウトにこだわっているらしい.
すでに説明したように技術文書や学術論文では図表の配置や改行などにあまりこだわりがない.
しかし, 不可能ではない. HTML では難しいが
(不可能ではないがHTMLでやるメリットが感じられないので対応する気がない),
PDF ではある程度のレイアウトの制御が可能である.

ただし, 本当に厳格なJIS準拠の組版にこだわるなら, LaTeX
を直接編集しなければならない.

\hypertarget{part-ux30c7ux30d0ux30c3ux30b0}{%
\part{デバッグ}\label{part-ux30c7ux30d0ux30c3ux30b0}}

残念ながら, 現状 \texttt{bookdown}
は完全にプログラミング知識のないエンドユーザでも縦横無尽に使用できるかと言うと,
まだまだ不安定でそのレベルには達していない. さらに悪いことに,
\texttt{rmarkdown} および \texttt{bookdown} は \texttt{knitr},
\texttt{pandoc}, LaTeX
といった様々なプログラムを継ぎ接ぎして実装されているため, R
の知識だけではエラーが起こった場合や,
意図したとおりの出力が得られないときに原因が分かりにくいことがある.
そこで, ここではエラーが出た際にどう対処するかのヒントを書いておく.

\hypertarget{ux88fdux672cux6642ux306eux30a8ux30e9ux30fcux3078ux306eux5bfeux51e6}{%
\chapter{製本時のエラーへの対処}\label{ux88fdux672cux6642ux306eux30a8ux30e9ux30fcux3078ux306eux5bfeux51e6}}

\hypertarget{ux30a8ux30e9ux30fcux304cux3069ux306eux30bfux30a4ux30dfux30f3ux30b0ux3067ux767aux751fux3057ux305fux304bux3092ux7279ux5b9aux3059ux308b}{%
\section{エラーがどのタイミングで発生したかを特定する}\label{ux30a8ux30e9ux30fcux304cux3069ux306eux30bfux30a4ux30dfux30f3ux30b0ux3067ux767aux751fux3057ux305fux304bux3092ux7279ux5b9aux3059ux308b}}

逆に言えば, \texttt{Rmd} ファイルを \texttt{md} ファイルに変換
(\texttt{knitr}による処理) するときにエラーが出たのか, \texttt{md}
を各ファイルに変換 \texttt{pandoc}
する際に起こったのかをまず特定するのが重要である. そのためには

\begin{enumerate}
\def\labelenumi{\arabic{enumi}.}
\tightlist
\item
  \texttt{keep\_md:\ true} を設定する
\item
  うまくいかないときはキャッシュを削除してから再実行する
\end{enumerate}

という対処法がある. (1) は文字通り中間出力ファイルである \texttt{.md}
を残すことを意味する. これが生成されないなら \texttt{knitr}
でのエラーだと分かるし, 中身を見て不自然な内容になっているのなら Rmd
の書き方が \texttt{knitr} に正しく評価されていないことがわかる.

キャッシュも私の経験上よくエラーの原因となっている.
以前に実行していたチャンクの結果が更新されていないせいで, \texttt{knitr}
の処理の不整合を起こすことがある. \texttt{*\_files}
には出力に必要な画像ファイルが, \texttt{*\_cache}
にはチャンク実行結果のキャッシュが残っている. 後者は
\texttt{knitr::opts\_chunk\$set(cache\ =\ T)}
などでキャッシュを残す設定にできるので, \texttt{F}
に設定した上でこれらのファイルを削除する.

処理に時間がかかるチャンクがあってキャッシュを作りたい場合は, 別途
\texttt{rds} や \texttt{RData} ファイルに結果を保存するという方法もある.
しかしもしプログラムの再現性を重視する場合,
この方法は望ましくないだろう. しかし残念ながら現状はこうするか,
ひたすら長い時間を待つしかない.

TODO: \url{https://bookdown.org/yihui/rmarkdown-cookbook/cache.html}

\hypertarget{yaml-ux30d5ux30edux30f3ux30c8ux30deux30bfux30fcux3092ux78baux8a8dux3059ux308b}{%
\section{YAML
フロントマターを確認する}\label{yaml-ux30d5ux30edux30f3ux30c8ux30deux30bfux30fcux3092ux78baux8a8dux3059ux308b}}

以前『\href{https://ill-identified.hatenablog.com/entry/2020/09/05/202403}{{[}R{]}
R Markdown の YAML
ヘッダでハマったおまえのための記事}』というブログ記事にも書いたように,
YAML フロントマターは慣れないと書き間違えやすいのが現状である.
もし自分で変更したのなら, 改めて確認すべきだろう. 特に,
製本直後にすぐに, 心当たりのないRプログラム関係のエラーが出る場合,
\textbf{チャンクではなく YAML
フロントマターの読み取りに失敗している可能性}がある.

以下の\textbf{4原則}を覚えておこう. 以前は \texttt{bookdown}
の話を想定してなかったので, さらに条文を1つ加えた.

\begin{enumerate}
\def\labelenumi{\arabic{enumi}.}
\tightlist
\item
  \texttt{output:} 以下はフォーマット関数への引数
\item
  トップレベルのオプションは \texttt{pandoc} のオプション
\item
  タイプミスや位置間違えでも\textbf{動いたり, 動かなかったりする}
\item
  \texttt{\_output.yml} および \texttt{\_bookdown.yml} を見る.
\end{enumerate}

\texttt{output:} には bookdown::gitbook など, \texttt{bookdown}
で提供されているフォーマット関数を指定しており,
その配下に記入するのはフォーマット関数に与える引数である. よって,
関数ヘルプを確認すれば有効な引数を知ることができる. しかし一方で,
\texttt{...} が引数になっていることがあるので,
タイプミスしてもエラーが出ないことがある.

また, YAMLの構文でサポートされている配列は誤評価を引き起こすことがある.

\begin{Shaded}
\begin{Highlighting}[]
\FunctionTok{output}\KeywordTok{:}
\AttributeTok{  bookdown:}\FunctionTok{:gitbook}\KeywordTok{:}
\AttributeTok{    }\FunctionTok{toc\_depth}\KeywordTok{:}\AttributeTok{ }\DecValTok{3}
\AttributeTok{    }\FunctionTok{toc}\KeywordTok{:}\AttributeTok{ }\CharTok{true}
\end{Highlighting}
\end{Shaded}

\begin{Shaded}
\begin{Highlighting}[]
\FunctionTok{output}\KeywordTok{:}
\AttributeTok{  bookdown:}\FunctionTok{:gitbook}\KeywordTok{:}
\AttributeTok{    }\KeywordTok{{-}}\AttributeTok{ }\FunctionTok{toc\_depth}\KeywordTok{:}\AttributeTok{ }\DecValTok{3}
\AttributeTok{    }\KeywordTok{{-}}\AttributeTok{ }\FunctionTok{toc}\KeywordTok{:}\AttributeTok{ }\CharTok{true}
\end{Highlighting}
\end{Shaded}

上の例は正しい記法である. 一方でハイフン \texttt{-} は YAML
では配列を記述するために用意されている. 下記の場合,
キーワード引数ではなく位置引数のような扱いになるため, \texttt{toc}
に対して \texttt{3} を代入することになり, エラーが発生する. 逆に言えば,
\texttt{-} を使う場合,
キーワードを書かずに値だけを正しい順番で書けば機能する.

インデントしないトップレベルの引数は, 基本的に \texttt{pandoc}
に与える引数である.
これ意味のない引数を与えてもエラーを返さないことが多いので,
タイプミスに注意する.

しかし, フォーマット関数に \texttt{pandoc\_args}
という構文をサポートしていることや, フォーマット関数で \texttt{pandoc}
の同名の引数を上書きする仕様のフォーマットもあるため,
上記は絶対ではない. これが原因で, 「\texttt{output:}
以下に書くべきものを間違えてトップレベルに書いたが,
意図したとおりに機能した」あるいはその逆が発生することがある. また,
\textbf{\texttt{pandoc}
の構文ではキーワードにハイフンを使うことができる}が,
フォーマットはRの関数でもあるため, ハイフンを使えず,
アンダースコアで置き換えられる. この違いも書き間違えの原因になる.

\hypertarget{pdf-ux306eux751fux6210ux306bux5931ux6557ux3059ux308bux5834ux5408}{%
\section{PDF
の生成に失敗する場合}\label{pdf-ux306eux751fux6210ux306bux5931ux6557ux3059ux308bux5834ux5408}}

それでもエラーが出る場合, 私の経験上ほとんどが生成した \texttt{.tex}
ファイルをタイプセットする際にエラーが発生している. \texttt{html}
との両立を考えると, どうしても \texttt{pandoc}
が解釈できる構文に限界がくるためである.

\begin{verbatim}
! LaTeX Error: XXXXX
\end{verbatim}

とか

\begin{verbatim}
Error: LaTeX failed to XXXX
\end{verbatim}

といったメッセージが表示されるのですぐ分かる. さらに丁寧なことに,
\texttt{tinytex}
のデバッグ方法への\href{https://yihui.org/tinytex/r/\#debugging}{リンク}まで表示される

この場合最も重要なのは, 以下に尽きる.

1 \texttt{options(tinytex.verbose\ =\ TRUE)} を設定する 2
\texttt{keep\_tex:\ true} を設定する

これは \texttt{keep\_md} と同様に, 中間ファイルである \texttt{.tex}
を残すことを意味する.

それでも解決しない場合,
改めてこのファイルを手動でタイプセットするのも1つの方法だ.
もしうまくいったり, 異なるエラーが出るのなら,
環境の違いが問題かもしれない. そして upBibTeX を使うのなら,
後者が唯一のデバッグ方法だ.

\hypertarget{appendix-ux88dcux907a}{%
\appendix}


ここでは \texttt{rmdja} の内部処理を解説する. \texttt{knitr} や
\texttt{rmarkdown} の仕様に精通している,
自分で細かい設定をしたいユーザ向けの解説である.

\hypertarget{ux30c7ux30d5ux30a9ux30ebux30c8ux5024ux306eux81eaux52d5ux8abfux6574}{%
\chapter{デフォルト値の自動調整}\label{ux30c7ux30d5ux30a9ux30ebux30c8ux5024ux306eux81eaux52d5ux8abfux6574}}

R Markdown で日本語文書を作成する上での大きな障害の1つである, YAML
フロントマターの設定を改善している. \texttt{rmdja}
の文書フォーマットはYAMLフロントマターのデフォルト値などを日本語文書に適したものに変更している.
さらに,
ユーザーをOSごとのフォントの違いや煩雑で重複だらけの設定から解放するため,
内部処理でも動的に設定変更している. もちろんこれらは ユーザーによる YAML
フロントマターやチャンクオプションの変更で上書きできる.

\hypertarget{ux30c7ux30d5ux30a9ux30ebux30c8ux306eux30d5ux30a9ux30f3ux30c8}{%
\section{デフォルトのフォント}\label{ux30c7ux30d5ux30a9ux30ebux30c8ux306eux30d5ux30a9ux30f3ux30c8}}

PDF 出力時のデフォルトフォントは, 生成時に OS を判定して設定している.
その設定は表 \ref{tab:font-default} のようなルールである.

\begin{table}

\caption{\label{tab:font-default}OS/エンジン別のデフォルトフォント}
\centering
\begin{tabular}[t]{>{}lllll}
\toprule
engine & Linux & Mac & Windows (>= 8) & Windows (それ以前)\\
\midrule
\cellcolor{gray}{XeLaTeX} & Noto & 游書体 & 游書体 & MSフォント\\
\cellcolor{gray}{LuaLaTeX} & Noto & ヒラギノ & 游書体 & MSフォント\\
\bottomrule
\end{tabular}
\end{table}

これらは \XeLaTeX ならば \texttt{zxjafont}, \LuaLaTeX ならば
\texttt{luatex-ja} で用意されているプリセットを使って設定している. 使用
OS の判定は R の基本関数による. なお, Noto フォントを選んだのは Ubuntu
18以降の日本語用フォントだからである. Ubuntu
から派生したOSにはプリインストールされていることが多いようだが, Debian,
Cent OS, Fedora 等にはおそらくプリインストールされていないので注意.
現時点ではフォントが\textbf{実際にインストールされているかを確認する機能はない}.

フォントのプリセットを指定した場合, 個別設定は無効になる. さらに,
3種類の和文フォントを全て設定していない場合もデフォルトのプリセットから選ばれる.

\hypertarget{ux30c1ux30e3ux30f3ux30afux306eux30c7ux30d5ux30a9ux30ebux30c8ux8a2dux5b9a}{%
\section{チャンクのデフォルト設定}\label{ux30c1ux30e3ux30f3ux30afux306eux30c7ux30d5ux30a9ux30ebux30c8ux8a2dux5b9a}}

デフォルトのグラフィックデバイスは, HTML では \texttt{PNG}, PDF では
\texttt{cairo\_pdf} としている. R
でよく描画するような単純な図形はベクタ画像が適しているが,
件数のとても多いデータの散布図などはベクタ画像にするとファイルサイズが大きくなるため,
そのような画像を適度に「劣化」させてファイルサイズを軽減してくれる
\texttt{cairo\_pdf} を標準としている. HTML に関しては,
そもそもデフォルトの設定でPDFが表示できないWebブラウザが多いことから,
PNG をデフォルトにした.

また, \texttt{block}, \texttt{block2}, \texttt{asis} などのブロックを
\texttt{echo=F} や \texttt{include=F} にするメリットはほぼないため,
\texttt{knitr::opts\_chunk\$set(echo\ =\ F,\ include\ =\ F)}
と一括設定してもこれらは \texttt{echo=T,\ include=T} のままである.
変更したい場合は, チャンクごとに設定することで有効になる.

\hypertarget{pdf-ux306eux7d44ux7248ux306bux95a2ux3059ux308bux7d30ux304bux3044ux8a71}{%
\chapter{PDF
の組版に関する細かい話}\label{pdf-ux306eux7d44ux7248ux306bux95a2ux3059ux308bux7d30ux304bux3044ux8a71}}

ここではpandocテンプレート等の設定を解説する.

3種類の和文フォントを個別設定をした場合, \XeLaTeX
はフォールバックフォントを有効にしている. \texttt{j****fontoptions}
以下に, \texttt{FallBack=...}
というオプションでフォールバックフォントを指定すれば有効になる.

用紙サイズは, デフォルトは \texttt{a4paper}, B5がよいなら
\texttt{b5paper} オプションを \texttt{classoptions:} に指定する.

PDF を印刷所に持ち込んだことがないため詳しいことはわからないが,
『\href{https://teastat.blogspot.com/2019/01/bookdown.html}{Bookdownによる技術系同人誌執筆}』で指摘されているようなトンボやノンブルは出力されるように作ってある
(そしてここで紹介されているようなLaTeXのコマンドの多くは \texttt{rmdja}
では書く必要がなくなった).

TODO: PART の扉ページにはまだノンブルが表示されない

\hypertarget{ux753bux50cfux306eux914dux7f6e}{%
\section{画像の配置}\label{ux753bux50cfux306eux914dux7f6e}}

現在,
PDFで画像の配置を固定する方法について何も特別なものを用意していない.
単純に自分は必要だとおもったことがないため. 固定したい場合は R Markdown
や Bookdown のドキュメントを参考にしてほしい. ただし,
通常は章や部をまたいで表示されることはない (はず).

\hypertarget{ux53d6ux308aux6d88ux3057ux7dda}{%
\section{取り消し線}\label{ux53d6ux308aux6d88ux3057ux7dda}}

LaTeX の各パッケージのバージョンによっては, 和文に取り消し線
(\texttt{\textbackslash{}sout})
を与えるとタイプセット時にエラーが出ることがある.
もともと\texttt{ulem.sty}は欧文を前提にしたものなので適当に妥協してほしい.

\hypertarget{todo-ux3057ux304bux3057ux82f1ux6587ux3067ux66f8ux304dux305fux3044ux5834ux5408}{%
\section{TODO:
しかし英文で書きたい場合}\label{todo-ux3057ux304bux3057ux82f1ux6587ux3067ux66f8ux304dux305fux3044ux5834ux5408}}

\texttt{rmdja} の機能を使いたいが,
執筆は英語でしたいと言う場合は最低限以下のような設定変更が必要である.
デフォルトは日本語用文書クラスのため,

\texttt{documentclass:\ book\ /\ report\ /\ article}

など欧文用文書クラスを指定する.

\texttt{rmdja} では和文フォントを参照するので,
和文フォントの設定も手動で解除する必要がある.

を指定する. そして各種見出しも英文用に調整する.

TODO

\hypertarget{jecon.bst-ux306eux7d39ux4ecb}{%
\chapter{jecon.bst の紹介}\label{jecon.bst-ux306eux7d39ux4ecb}}

和文と欧文を使い分けたスタイルファイルとして, \texttt{jecon.bst} がある.
\texttt{jecon.bst} の公式ではなく,
私がカスタマイズしたバージョンでも良い.
こちらは本来よりも電子媒体としての利用を重視して,

\begin{itemize}
\tightlist
\item
  参照URLを表示せず, ハイパーリンクのみにする
\item
  ArXiv ID の表示とハイパーリンク追加
\end{itemize}

といった変更をしている. 後者は, BibTeX エントリに以下のように
\texttt{archivePrefix} に \texttt{arxiv} と言う値が入っていると,
\texttt{eprint} の値が ArXiv ID として表示される. これは ArXiv から直接
\texttt{.bib} ファイルを取得したり, Zotero
などでインポートすれば必ず入力される項目である.

\begin{Shaded}
\begin{Highlighting}[]
\CommentTok{archivePrefix = \{arXiv\},}
\CommentTok{eprint = \{XXXX.YYYYY\},}
\CommentTok{...}
\end{Highlighting}
\end{Shaded}

TODO: 現在 jecon.bst の表示も少しおかしいので確認中.

\chapter*{参考文献}

\printbibliography

\end{document}
